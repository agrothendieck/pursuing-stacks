%Begin











\chapter{Take-off}
\label{ch:I}
%\addcontentsline{toc}{chapter}{The Take-off (a letter to Daniel Quillen)}

\par\hfill Les Aumettes \ondate{19.2.}1983\namedpspage{L.1}{1}\par

Dear Daniel\scrcomment{Daniel Quillen},

% 1
\hangsection{The importance of innocence.}\label{sec:1}%
Last year Ronnie Brown from Bangor\scrcomment{cf.\ \textcite{Kunzer2015}} sent me a heap of reprints and
preprints by him and a group of friends, on various foundational
matters of homotopical algebra. I did not really dig through any of
this, as I kind of lost contact with the technicalities of this kind
(I was never too familiar with the homotopy techniques anyhow, I
confess) -- but this reminded me of a few letters I had exchanged with
Larry Breen in 1975\footnote{These letters are reproduced as an 
``appendix'' at the end of this chapter.}, where I had developed an outline of a program for
a kind of ``topological algebra'', viewed as a synthesis of
homotopical and homological algebra, with special emphasis on topoi --
most of the basic intuitions in this program arising from various
backgrounds in algebraic geometry. Some of those intuitions we
discussed, I believe, at IHES eight or nine years before\scrcomment{\textcite{tapisQ}}, then, at a time
when you had just written up your nice ideas on axiomatic homotopical
algebra,\scrcomment{\textcite{Quillen1967}} published since in
Springer's Lecture Notes. I write you under the assumption that you
have not entirely lost interest for those foundational questions you
were looking at more than fifteen years ago. One thing which strikes
me, is that (as far as I know) there has not been any substantial
progress since -- it looks to me that an understanding of the basic
structures underlying homotopy theory, or even homological algebra
only, is still lacking -- probably because the few people who have a
wide enough background and perspective enabling them to feel the main
questions, are devoting their energies to things which seem more
directly rewarding. Maybe even a wind of disrepute for any
foundational matters whatever is blowing nowadays\footnote{When making this suggestion about there being a ``wind of disrepute for any foundational matters whatever'', I little suspected that the former friend to whom I was communicating my ponderings as they came, would take care of providing a most unexpected confirmation. As a matter of fact, this letter never got an answer, nor was it even read! Upon my inquiry nearly one year later, this colleague appeared sincerely surprised that I could have expected even for a minute that he might possibly read a letter of mine on mathematical matters, well knowing the kind of ``general nonsense'' mathematics that was to be expected from me\dots}!  In this respect,
what seems to me even more striking than the lack of proper
foundations for homological and homotopical algebra, is the absence I
daresay of proper foundations for topology itself! I am thinking here
mainly of the development of a context of ``tame'' topology, which (I
am convinced) would have on the everyday technique of geometric
topology (I use this expression in contrast to the topology of use for
analysts) a comparable impact or even a greater one, than the
introduction of the point of view of schemes had on algebraic
geometry\footnote{For some particulars about a program of ``tame topology'', I refer to ``Esquisse d'un Programme'', sections 5 and 6, which is included in Réflexions Mathématiques 1.}. The psychological drawback here I believe is not anything
like messiness, as for homological and homotopical algebra (or for
schemes), but merely the inrooted inertia which prevents us so
stubbornly from looking innocently, with fresh eyes, upon things,
without being dulled and imprisoned by standing habits of thought,
going with a familiar context -- \emph{too} familiar a context! The
task of working out the foundations of tame topology, and a
corresponding structure theory for ``stratified (tame) spaces'', seems
to me a lot more urgent and exciting still
than\namedpspage{L.1prime}{1'} any program of homological,
homotopical or topological algebra.

% 2
\hangsection{A short look into purgatory.}\label{sec:2}%
The motivation for this letter was the latter topic
however. Ronnie Brown and his friends are competent algebraists and
apparently strongly motivated for investing energy in foundational
work, on the other hand they visibly are lacking the necessary scope
of vision which geometry alone provides\footnote{I have to apologise for this rash statement, as later correspondence made me realise that ``Ronnie Brown and his friends'' do have stronger contact with ``geometry'' than I suspected, even though they are not too familiar with algebraic geometry!}. They seem to me kind of
isolated, partly due I guess to the disrepute I mentioned before -- I
suggested to try and have contact with people such as yourself, Larry
Breen, Illusie and others, who have the geometric insight and who
moreover, may not think themselves too good for indulging in
occasional reflection on foundational matters and in the process help
others do the work which should be done.

At first sight it has seemed to me that the Bangor group\footnote{The ``Bangor group'' is made up by Ronnie Brown and Tim Porter as the two fixed points, and a number of devoted research students. Moreover Ronnie Brown is working in close contact with J.L. Loday and J. Pradines in France.} had indeed
come to work out (quite independently) one basic intuition of the
program I had envisioned in those letters to Larry Breen -- namely
that the study of $n$-truncated homotopy types (of semisimplicial
sets, or of topological spaces) was essentially equivalent to the
study of so-called $n$-groupoids (where $n$ is any natural
integer). This is expected to be achieved by associating to any space
(say) $X$ its ``fundamental $n$-groupoid'' $\Pi_n(X)$, generalizing
the familiar Poincar\'e fundamental groupoid for $n=1$. The obvious
idea is that $0$-objects of $\Pi_n(X)$ should be the points of $X$,
$1$-objects should be ``homotopies'' or paths between points,
$2$-objects should be homotopies between $1$-objects, etc. This
$\Pi_n(X)$ should embody the $n$-truncated homotopy type of $X$, in
much the same way as for $n=1$ the usual fundamental groupoid embodies
the $1$-truncated homotopy type. For two spaces $X,Y$, the set of
homotopy-classes of maps $X\to Y$ (more correctly, for general $X,Y$,
the maps of $X$ into $Y$ in the homotopy category) should correspond
to $n$-equivalence classes of $n$-functors from $\Pi_n(X)$ to
$\Pi_n(Y)$ -- etc. There are very strong suggestions for a nice
formalism including a notion of geometric realization of an
$n$-groupoid, which should imply that any $n$-groupoid (or more
generally of an $n$-category) is relativized over an arbitrary topos
to the notion of an $n$-gerbe (or more generally, an $n$-stack), these
become the natural ``coefficients'' for a formalism of non-commutative
cohomological algebra, in the spirit of Giraud's thesis.

But all this kind of thing for the time being is pure heuristics -- I
never so far sat down to try to make explicit at least a definition of
$n$-categories and $n$-groupoids, of $n$-functors between these
etc. When I got the Bangor reprints I at once had the feeling that
this kind of work\namedpspage{L.2}{2} had been done and the homotopy category
expressed in terms of \oo-groupoids. But finally it appears this is
not so, they have been working throughout with a notion of
\oo-groupoid too restrictive for the purposes I had in mind (probably
because they insist I guess on strict associativity of compositions,
rather than associativity up to a (given) isomorphism, or rather,
homotopy) -- to the effect that the simply connected homotopy types
they obtain are merely products of Eilenberg-MacLane spaces, too bad!
They do not seem to have realized yet that this makes their set-up
wholly inadequate to a sweeping foundational set-up for homotopy. This
brings to the fore again to work out the suitable definitions for
$n$-groupoids -- if this is not done yet anywhere. I spent the
afternoon today trying to figure out a reasonable definition, to get a
feeling at least of where the difficulties are, if any. I am guided
mainly of course by the topological interpretation. It will be short
enough to say how far I got. The main part of the structure it seems
is expressed by the sets $F_i$ ($i\in\bN$) of $i$-objects, the source,
target and identity maps
\begin{align*}
  \cst s_1^i, \cst t_1^i &: F_i \to F_{i-1} \quad (i\ge 1) \\
  \cst k_1^i &: F_i \to F_{i+1} \quad (i\in\bN) \\
\intertext{and the symmetry map (passage to the inverse)}
  \inv_i &: F_i \to F_i \quad (i\ge 1),
\end{align*}
satisfying\marginpar{\emph{notation:}\\ $d_a=\cst k_1^i(a)$,\\ $\check u =
  \inv_i(u)$} some obvious relations: $\cst k_1^i$ is right inverse to
the source and target maps $\cst s_1^{i+1}, \cst t_1^{i+1}$, $\inv_i$
is an involution and ``exchanges'' source and target, and moreover for
$i\ge2$
\begin{align*}
  \cst s_1^{i-1} \cst s_1^i
  &= \cst s_1^{i-1} \cst t_1^i \;
    \Bigl(\eqdef \cst s_2^i : F_i \to F_{i-2} \Bigr) \\
  \cst t_1^{i-1} \cst s_1^i
  &= \cst t_1^{i-1} \cst t_1^i \;
    \Bigl(\eqdef \cst t_2^i : F_i \to F_{i-2} \Bigr) ;
\end{align*}
thus the composition of the source and target maps yields, for $0\le
j\le i$, just \emph{two} maps
\begin{equation*}
  \cst s_\ell^i, \cst t_\ell^i : F_i \to F_{i-\ell} = F_j \quad (\ell = i-j).
\end{equation*}
The next basic structure is the composition structure, where the
usual composition of arrows, more specifically of $i$-objects
($i\ge1$) $v \circ u$ (defined when $\cst t_1(u)=\cst s_1(v)$) must be
supplemented by the Godement-type operations $\mu * \lambda$ when
$\mu$ and $\lambda$ are ``arrows between arrows'', etc. Following this
line of thought, one gets the composition maps
\begin{equation*}
  (u,v) \mapsto (v *_{\ell}^i u) : (F_i, \cst s_\ell^i) \times_{F_{i-\ell}}
  (F_i,\cst s_\ell^i) \to F_i,
\end{equation*}
the $*_\ell$-composition of $i$-objects ($1\le \ell\le i$) being defined
when the $\ell$-target of $u$ is equal to the $\ell$-source of $v$,
and then we have
\begin{equation*}
  \left.\begin{aligned}
    \cst s_1^i(v *_\ell u) &=  \cst s_1^i(v) *_{\ell-1} \cst s_1^i(u) \\
    \cst t_1^i(v *_\ell u) &=  \cst t_1^i(v) *_{\ell-1} \cst s_1^i(u)
    \end{aligned}\right\}
  \quad \text{$\ell\ge2$ i.e. $\ell-1\ge1$}
\end{equation*}
and\namedpspage{L.2prime}{2'} for $\ell=1$
\begin{align*}
  \cst s_1(v *_1 u) &= s_1(u) \\
  \cst t_1(v *_1 u) &= t_1(v)
\end{align*}
(NB\enspace the operation $v *_1 u$ is just the usual composition $v \circ
u$).

One may be tempted to think that the preceding data exhaust the
structure of \oo-groupoids, and that they will have to be supplemented
only by a handful of suitable \emph{axioms}, one being
\emph{associativity} for the operation
$\overset{i}{\underset{\ell}{*}}$, which can be expressed essentially
by saying that that composition operation turns $F_i$ into the set of
arrows of a category having $F_{i-\ell}$ as a set of objects (with the
source and target maps $\cst s_\ell^i$ and $\cst t_\ell^i$, and with
identity map $\cst k_\ell^{i-\ell} : F_{i-\ell} \to F_i$ the
composition of the identity maps $F_{i-\ell} \to F_{i-\ell+1} \to
\dots \to F_{i-1} \to F_i$), and another being the Godement relation
\begin{equation*}
  (v' *_\alpha v) *_\nu (u' *_\alpha u) = (v' *_\nu u') *_\alpha (v
  *_\nu u)
\end{equation*}
(with the assumptions $1\le\alpha\le\nu$, and $u,u',v,v' \in F_i$ and
\begin{equation*}
  \left\{\begin{aligned}
      \cst t_\alpha(u)=\cst s_\alpha(u') \\
      \cst t_\alpha(v)=\cst s_\alpha(v') \\
    \end{aligned}\right.\quad
  \cst t_\nu(u) = \cst s_\nu(v) = \cst s_\nu(v') = \cst t_\nu(u')
\end{equation*}
expressing that both members are defined), plus the two relations
concerning the inversion of $i$-objects ($i\ge1$) $u \mapsto \check
u$,
\begin{multline*}
  u *_1 \check u = \id_{\cst t_1(u)} , \qquad \check u *_1 u =
  \id_{\cst s_1(u)} , \qquad
  (\check v *_\ell \check u) = \text{?} \quad(\ell\ge2)
\end{multline*}
It just occurs to me, by the way, that the previous description of
basic (or ``primary'') data for an \oo-groupoid is already incomplete
in some rather obvious respect, namely that the symmetry-operation
$\inv_i : u \mapsto \check u$ on $F_i$ must be complemented by $i-1$
similar involutions on $F_i$, which corresponds algebraically to the
intuition that when we have an $(i+1)$-arrow $\lambda$ say between two
$i$-arrows $u$ and $v$, then we must be able to deduce from it another
arrow from $\check u$ to $\check v$ (namely $u\mapsto\check u$ has a
``functorial character'' for variable $u$)? This seems a rather
anodine modification of the previous set-up, and is irrelevant for the
main point I want to make here, namely: that for the notion of
\oo-groupoids we are after, all the equalities just envisioned in this
paragraph (and those I guess which will ensure naturality by the
necessary extension of the basic involution on $F_i$) should be
replaced by ``homotopies'', namely by $(i+1)$-arrows between the two
members. These arrows should be viewed, I believe, as being part of
the data, they appear here as a kind of ``secondary'' structure. The
difficulty which appears now is to work out the natural coherence
properties\namedpspage{L.3}{3} concerning this secondary structure. The first thing I
could think of is the ``pentagon axiom'' for the associativity data,
which occurs when looking at associativities for the compositum (for
$\overset{i}{\underset{\ell}{*}}$ say) of four factors. Here again the
first reflex would be to write down, as usual, an \emph{equality} for
two compositions of associativity isomorphisms, exhibited in the
pentagon diagram. One suspects however that such equality should,
again, be replaced by a ``homotopy''-arrow, which now appears as a
kind of ``ternary'' structure -- before even having exhausted the list
of coherence ``relations'' one could think of with the respect to the
secondary structure! Here one seems caught at first sight in an
infinite chain of ever ``higher'', and presumably, messier structures,
where one is going to get hopelessly lost, unless one discovers some
simple guiding principle for shedding some clarity into the mess.

% 3
\hangsection{``Fundamental \texorpdfstring{\oo}{oo}-groupoids'' as
  objects of a ``model category''?}\label{sec:3}%
I thought of writing you mainly because I believe that,
if anybody, you should know if the kind of structure I am looking for
has been worked out -- maybe even \emph{you} did? In this respect, I
vaguely remember that you had a description of $n$-categories in terms
of $n$-semisimplicial sets, satisfying certain exactness conditions,
in much the same way as an ordinary category can be interpreted, via
its ``nerve'', as a particular type of semisimplicial set. But I have
no idea if your definition applied only for describing $n$-categories
with strict associativities, or not\footnote{Definitely only for \emph{strict} associativity.}.

Still some contents in the spirit of your axiomatics of homotopical
algebra -- in order to make the question I am proposing more seducing
maybe to you! One comment is that presumably, the category of
\oo-groupoids (which is still to be defined) is a ``model category''
for the usual homotopy category; this would be at any rate one
plausible way to make explicit the intuition referred to before, that
a homotopy type is ``essentially the same'' as an \oo-groupoid up to
\oo-equivalence. The other comment: the construction of the
fundamental \oo-groupoid of a space, disregarding for the time being
the question of working out in full the pertinent structure on this
messy object, can be paraphrased in any model category in your sense,
and yields a functor from this category to the category of
\oo-groupoids, and hence (by geometric realization, or by
localization) also to the usual homotopy category\footnote{This idea is taken up again in section 12. The statement made here is a little rash, as for existence and uniqueness (in a suitable sense) of this functor. Compare note $(^{17})$ below.}. Was this functor
obvious beforehand? It is of a non-trivial nature only when the model
category is \emph{not} pointed -- as a matter of fact the whole
construction can be carried out canonically, in terms of a ``cylinder
object'' $I$ for the final object $e$ of the model category, playing
the role of the unit segment.\namedpspage{L.3prime}{3'}
It's high time to stop this letter -- please excuse me if it should
come ten or fifteen years too late, or maybe one year too early. If
you are not interested for the time being in such general nonsense,
maybe you know someone who is \ldots

\bigskip

Very cordially yours

\bigskip

\hfill Alexander

\newpage

\presectionfill\ondate{20.2.}\namedpspage{L.4}{4}\par

I finally went on pondering about a definition of
\oo-groupoids, and it seems to me that, after all, the topological
motivation does furnish the ``simple guiding principle'' which
yesterday seemed to me to be still to be discovered, in order not to
get lost in the messiness of ever higher order structures. Let me try
to put it down roughly.

% 4
\hangsection{A bit of ordering in the mess of ``higher order
  structures''.}\label{sec:4}%
First I would like to correct somewhat the rather indiscriminate
description I gave yesterday of what I thought of viewing as
``primary'', secondary, ternary etc.\ structures for an
\oo-groupoid. More careful reflection conduces to view as the most
primitive, starting structure on the set of sets $F_i$ ($i\in\bN$), as
a skeleton on which progressively organs and flesh will be added, the
mere data of the source and target maps
\begin{equation*}
  \cst s_i, \cst t_i : F_i \rightrightarrows F_{i-1}
  \quad(i \ge 1),
\end{equation*}
which it will be convenient to supplement formally by corresponding
maps $\cst s_0, \cst t_0$ for $i=1$, from $F_0$ to
$F_{-1} \eqdef \text{one-element set}$. In a moment we will pass to a
universal situation, when the $F_i$ are replaced by the corresponding
``universal'' objects $\boldsymbol F_i$ in a suitable category stable
under finite products, where $F_{-1}$ will be the final element. For
several reasons, it is not proper to view the inversion maps
$\inv_i : F_i \to F_i$, and still less the other $i-1$ involutions on
$F_i$ which I at first overlooked, as being part of the primitive or
``skeletal'' structure. One main reason is that already for the most
usual $2$-groupoids, such as the $2$-groupoid whose $0$-objects are
ordinary ($1$-)groupoids, the $1$-objects being \emph{equivalences} between
these (namely functors which are fully faithful and essentially
surjective), and the $2$-objects morphisms (or ``natural
transformations'') between such, there is \emph{not}, for an
$i$-object $f: C \to C'$, a natural choice of an ``inverse'' namely of
a quasi-inverse in the usual sense. And even assuming that such
quasi-inverse is chosen for every $f$, it is by no means clear that
such choice can be made involutive, namely such that
$(f\upvee)\upvee = f$ for every $f$ (and not merely $(f\upvee)\upvee$
isomorphic to $f$). The maps $\inv_i$ will appear rather, quite
naturally, as ``primary structure'', and they will not be involutions,
but ``pseudo-involutions'' (namely involutions ``up to homotopy''). It
turns out that among the various functors that we will construct, from
the category of topological spaces to the category of \oo-groupoids
(the construction depending on arbitrary choices and yielding a large
bunch of mutually non-isomorphic functors, which however are
``equivalent'' in a sense which will have to be made precise) -- there
are choices neater than others, and some of these will yield in the
primary structure maps $\inv_i$ which are actual involutions and
similarly for the other pseudo-involutions, appearing in succession as
higher order structure. The possibility of such neat and fairly
natural choices had somewhat misled me\namedpspage{L.4prime}{4'} yesterday.

What may look less convincing though at first sight, is my choice to
view as non-primitive even the ``degeneration maps'' $\cst k_1^i : F_i
\to F_{i+1}$, associating to every $i$-objects the ($i+1$)-object acting
as an identity on the former. In all cases I have met so far, these
maps are either given beforehand with the structure (of a $1$-category
or $2$-category say), or they can be uniquely deduced from the
axioms. In the present set-up however, they seem to me to appear more
naturally as ``primary'' (not as primitive) structure, much in the
same way as the $\inv_i$. Different choices for associating an
\oo-groupoid to a topological space, while yielding the same base-sets
$F_i$, will however (according to this point of view) give rise to
different maps $\cst k_1^i$. The main motivation for this point of
view comes from the fact that the mechanism for a uniform construction
of the chain of ever higher order structures makes a basic use of the
source and target maps only and of the ``transposes'' (see below), and
(it seems to me) not at all of the degeneration maps, which in this
respect are rather confusing the real picture, if viewed as
``primitive''.
The degeneration maps rather appear as typical cases of primary
structure, probably of special significance in the practical handling
of \oo-groupoids, but not at all in the conceptual machinery leading
up to the construction of the structure species of ``\oo-groupoids''.

Much in the same way, the composition operations
$\overset{i}{\underset{1}{*}}$ are viewed as primary, not as
primitive or skeletal structure. Their description for the fundamental
\oo-groupoid of a space -- for instance the description of composition
of paths -- depends on arbitrary choices, such as the choice of an
isomorphism (say) between $(I,1) \amalg_e (I,0)$ and $I$, where $I$ is
the unit interval, much in the same way as the notion of an inverse of
a path depends on the choice of an isomorphism of $I$ with itself,
exchanging the two end-points $0$ and $1$. The operations
$\overset{i}{\underset{2}{*}}$ of Godement take sense only once the
composition operations $*_1$ are defined -- they are ``secondary
structure'', and successively the operations $*_3, \ldots, *_i$ appear
as ternary etc.\ structure. This is correctly suggested by the
notations which I chose yesterday, where however I hastily threw all
the operations into a same pot baptized ``primary structure''!

% 5
\hangsection{Jumping over the abyss!}\label{sec:5}%
It is about time though to come to a tentative precise definition of
description of the process of stepwise introduction of an increasing
chain of higher order structure. This will be done by introducing a
canonical sequence of categories and functors
\[ C_0 \to C_1 \to C_2 \to \dots \to C_n \to C_{n+1} \to \cdots,\]
where $C_n$ denotes the category harbouring the ``universal''\namedpspage{L.5}{5}
partial structure of a would-be \oo-groupoid, endowed only with its
``structure of order $\le n$''. The idea is to give a direct inductive
construction of this sequence, by describing $C_0$, and the passage
from $C_n$ to $C_{n+1}$ $(n\ge0$), namely from an $n$-ary to
$(n+1)$-ary structure. As for the meaning of ``universal structure'',
once a given structure species is at hand, it depends on the type of
categories (described by the exactness properties one is assuming for
these) one wants to take as carriers for the considered structure, and
the type of exactness properties one assumes for the functor one
allows between these. The choice depends partly on the particular
species; if it is an algebraic structure which can be described say by
a handful of composition laws between a bunch of base sets (or
base-\emph{objects}, when looking at ``realizations'' of the structure not
only in the category \Sets), one natural choice is to take categories
with finite products, and functors which commute to these. For more
sophisticated algebraic structures (including the structure of
category, groupoid or the like), which requires for the description of
data or axioms not only finite products, but also some fiber products,
one other familiar choice is to take categories with finite inverse
limits, and left exact functors. Still more sophisticated structures,
when the description of the structure in terms of base objects
requires not only some kind of inverse limits, but also more or less
arbitrary direct limits (such as the structure of a comodule over an
algebra, which requires consideration of tensor products over a ring
object\footnote{Another important example is the structure of a ``torsor'' under a group $G$ (torsor = principal homogeneous space). When this group $G$ is fixed, the corresponding classifying topos $\B_G$ is the natural purely algebraic substitute for the familiar ``classifying space'' for the discrete group $G$}.), correspondingly more
stringent exactness conditions will have to be imposed upon categories and
functors between these, for the structure to make a sense in these
categories, and the functors to transform a structure of this type in
one category into one of same type in another. In most examples I have
looked up, everything is OK taking categories which are topoi, and
functors between these which are inverse image functors for morphisms
of topoi, namely which are left exact and commute with arbitrary
direct limits. There is a general theorem for the existence of
universal structures\footnote{Such a theory was developed in a seminar I gave at Buffalo in 1973}, covering all these cases -- for instance there
is a ``classifying topos'' for most algebro-geometric structures,
whose cohomology say should be viewed as the ``classifying
cohomology'' of the structure species considered. In the case we are
interested in here, it is convenient however to work with the smallest
categories $C_n$ feasible -- which amounts to being as generous as
possible for the categories one is allowing as carriers for the
structure of an \oo-groupoid, and for the functors between these which
are expected to carry an \oo-groupoid into an \oo-groupoid. What we
will do is define ultimately a structure of an \oo-groupoid in a
category $C$, as a sequence of objects $F_i$ ($i\in\bN$), endowed with
some structure to be defined,\namedpspage{L.5prime}{5'} assuming merely that in $C$
finite products of the $F_i$ exist, plus certain finite inverse limits
built up with the $F_i$'s and the maps $\cst s_\ell^i$, $\cst
t_\ell^i$ between them (the iterated source and target maps). It
should be noted that the type of $\varprojlim$ we allow, which will
have to be made precise below, is fixed beforehand in terms of the
``skeletal'' or ``primitive'' structure alone, embodied by the family
of couples $(\cst s_i, \cst t_i)_{i\in\bN}$. This implies that the
categories $\bC_i$ can be viewed as having \emph{the same set of
  objects}, namely the objects $\boldsymbol F_i$ (written
in bold\scrcomment{was: \emph{underlined}} now to indicate their
universal nature, and including as was said before $\boldsymbol F_{-1} =
\text{final object}$), plus the finite products and iterated fiber
products of so-called ``standard'' type. While I am writing, it
appears to me even that the finite products here are of no use (so we
just drop them both in the condition on categories which are accepted
for harbouring \oo-groupoids, and in the set of objects of the
categories $C_i$). Finally, the common set of objects of the
categories $\bC_i$ is the set of ``standard'' iterated fiber
products of the $\boldsymbol F_i$, built up using only the primitive
structure embodied by the maps $\cst s_i$ and $\cst
t_i$ (which I renounce henceforth to underline!).
This at the same time gives, in principle, a precise definition of
$\bC_0$, at least up to equivalence -- it should not be hard anyhow
to give a wholly explicit description of $\bC_0$ as a small
category, having a countable set of objects, once the basic notion of
the standard iterated fiber-products has been explained.

Once $\bC_0$ is constructed, we will get the higher order
categories $\bC_1$ (primary), $\bC_2$ etc.\ by an inductive
process of \emph{successively adding arrows}. The category $\bC_\oo$
will then be defined as the direct limit of the categories
$\bC_n$, having the same objects therefore as $\bC_0$, with
\[\Hom_\oo(X, Y) = \varinjlim_n \Hom_n(X, Y)\]
for any two objects. This being done, giving a structure of
\oo-groupoid in any category $C$, will amount to giving a functor
\[\bC_\oo \to C\]
commuting with the standard iterated fiber-products. This can be
reexpressed, as amounting to the same as to give a sequence of objects
($F_i$) in $C$, and maps $s_\ell^i, t_\ell^i$ between these,
satisfying the two relations I wrote down yesterday (page \hyperref[p:L.2]2) (and
which of course have to be taken into account when defining $C_0$ to
start with, I forgot to say before), and such that
``standard'' fiber-products (I'll drop the qualification ``iterated''
  henceforth!) defined in terms of these data should
exist in $C$, plus a bunch of maps between these fiber-products (in
fact, it will suffice to give such maps with target among the
$F_i$'s), satisfying certain relations embodied in the structure of
the category $\bC_\oo$. I am convinced that this bunch of maps
(namely the maps stemming from arrows in $\bC_\oo$) not only is
infinite, but cannot either be generated in the\namedpspage{L.6}{6} obvious
sense by a finite number, nor even by a finite number of infinite
series of maps such as $\cst k_\ell^i$, $\inv_i$,
$\overset{i}{\underset{\ell}{*}}$, the compatibility arrows in the
pentagon, and the like. More precisely still, I am convinced that none
of the functors $\bC_n \to \bC_{n+1}$ is an equivalence, which
amounts to saying that the structures of increasing order form a
strictly increasing sequence -- at every step, there is actual extra
structure added. This is perhaps evident beforehand to topologists in
the know, but I confess that for the time being it isn't to me, in
terms uniquely of the somewhat formal description I will make of the
passage of $\bC_n$ to $\bC_{n+1}$\footnote{That $\overset{i}{\underset{2}{*}}$ is of order $\ell$ is heuristically clear, but will require a proof none the less}. This step theoretically is all
that remains to be done, in order to achieve an explicit construction
of the structure species of an \oo-category (besides the definition of
standard fiber-products) -- without having to get involved, still less
lost, in the technical intricacies of ever messier diagrams to write
down, with increasing order of the structures to be added\ldots

% 6
\hangsection[The topological model: hemispheres building up the \dots]%
{The topological model: hemispheres building up the
  \texorpdfstring{\textup(}{(}tentative\texorpdfstring{\textup)}{)}
  ``universal \texorpdfstring{\oo-\textup(co\textup)}{oo-(co)}groupoid''.}%
\label{sec:6}%
In the outline of a method of construction for the structure species,
there has not been any explicit mention so far of the topological
motivation behind the whole approach, which could wrongly give the
impression of being a purely algebraic one. However, topological
considerations alone are giving me the clue both for the description
of the so-called standard fiber products, and of the inductive step
allowing to wind up from $\bC_n$ to $\bC_{n+1}$ ! The heuristics
indeed of the present approach is simple enough, and suggested by the
starting task, to define pertinent structure on the system of sets
$F_i(X)$ of ``homotopies'' of arbitrary order, associated to an
arbitrary topological space. In effect, the covariant functors
\[ X \mapsto F_i(X) : \Spaces \to \Sets
\]
are representable by spaces $D_i$, which are easily seen to be
$i$-cells. The source and target maps $s_\ell^i, t_\ell^i :
F_i(X) \rightrightarrows F_{i-1}(X)$ are transposed to maps, which I may
denote by the same letters
\[s_\ell^i, t_\ell^i : D_{i-1} \rightrightarrows D_i.\]
Handling around a little, one easily convinces oneself that all the
main structural items on $F_*(X)$ which one is figuring out in
succession, such as the degeneracy maps $k_\ell^i$, the inversion maps
$\inv_i$, the composition $v\cdot u = v *_i u$ for $i$-objects, etc.,
are all transposed of similar maps which are defined between the cells
$D_i$, or which go from such cells to certain spaces, deduced from
these by gluing them together -- the most evident example in this
respect being the composition of paths, which is transposed of a map
from the unit segment $I$ into $(I,1)\amalg_e (I,0)$, having
preassigned values on the endpoints of $I$ (which correspond in fact
to the images of the two maps $s_1^1,t_1^1 : D_0
= \text{one point} \rightrightarrows D_1 =
I$).\namedpspage{L.6prime}{6'} 

In a more suggestive way, we could say from this
experiment that the family of discs $(D_i)_{i\in\bN}$, together with
the maps $s,t$ and a lot of extra structure which enters into the
picture step by step, is what we would like to call a
\emph{co-\oo-groupoid in the category} \cTop{} of topological spaces
(namely an \oo-groupoid in the dual category $\cTop\op$), and
that the structure of \oo-category on $F_*(X)$ we want to describe is
the transform of this co-structure into an \oo-groupoid, by the
contravariant functor from \cTop{} to \Sets{} defined by $X$. The
(iterated) amalgamated sums in \cTop{} which allow to glue together the
various $D_i$'s using the $s$ and $t$ maps between them, namely the
corresponding fibered products in $\cTop\op$, are indeed
transformed by the functor $h_X$ into fibered products of \Sets.
The suggestion is, moreover, that if we view our co-structure in \cTop{}
as a co-structure in the subcategory of Top, say $B_\oo$, whose
objects are the cells $D_i$ and the amalgamated sums built up with
these which step-wise enter into play, and whose arrows are all those
arrows which are introduced step-wise to define the co-structure, and
all compositions of these -- that this should be \emph{the universal
  structure} in the sense dual to the one we have been contemplating
before; or what amount to the same, that the corresponding
\oo-groupoid structure in the dual category $B_\oo\op$ is
``universal'' -- which means essentially that it is none other than
the universal structure in the category $\bC_\oo$ we are
after. Whether or not this expectation will turn out to be correct (I
believe it is\footnote{(Added \alsoondate{23.2.83} I don't believe it now any more - and I do not really care - compare comments in section 11}), we should be aware that, while the successive
introduction of maps between the cells $D_i$ and their ``standard''
amalgamated sums (which we will define precisely below) depends at
every stage on arbitrary choices, the categories $\bC_n$ and their
limit $\bC_\oo$ do not depend on any of these choices; assuming the
expectation is correct, this means that up to (unique) isomorphism,
the category $B_\oo$ (and each of the categories $B_n$ of which it
appears as the direct limit) is independent of those choices -- the
isomorphism between two such categories transforming any one choice
made for the first, into the corresponding choice made for the
second. Also, while this expectation was of course the crucial
motivation leading to the explicit description of $\bC_0$ and of
the inductive step from $\bC_n$ to $\bC_{n+1}$, this description
seems to me a reasonable one and in any case it makes a formal sense,
quite independently of whether the expectation proves valid or not.

% 7
\hangsection{Gluing hemispheres: the ``standard amalgamations''.}%
\label{sec:7}%
Before pursuing, it is time to give a more complete description of the
primitive structure on $(D_i)$, as embodied by the maps $s,t$, which I
will now denote by\footnote{\emph{NB}\enspace it is more natural to consider
  $\varphi^+$ as ``target'' and $\varphi^-$ as ``source''.}
\[\varphi_i^+,\varphi_i^- :  D_i \rightrightarrows D_{i+1}. \]
It appears that these maps are injective, that their images make
up\namedpspage{L.7}{7} the boundary $S_i = \dot D_{i+1}$ of $D_{i+1}$, more
specifically these images are just two ``complementary'' hemispheres
in $S_i$, which I will denote by $S_i^+$ and $S_i^-$. The kernel of
the pair $(\varphi_i^+, \varphi_i^-)$ is just $S_{i-1} = \dot D_i$,
and the common restriction of the maps $\varphi_i^+, \varphi_i^-$ to
$S_{i-1}$ is an isomorphism
\[ S_{i-1} \simeq  S_i^+ \cap S_i^-.\]
This $S_{i-1}$ in turn decomposes into the two hemispheres $S_{i-1}^+,
S_{i-1}^-$, images of $D_{i-1}$. Replacing $D_{i+1}$ by $D_i$, we see
that the $i$-cell $D_i$ is decomposed into a union of $2i+1$ closed
cells, one being $D_i$ itself, the others being canonically isomorphic
to the cells $S_j^+,S_j^-$ ($0\le j\le i$), images of $D_j\to D_n$ by
the iterated morphisms
\[\varphi_{n,j}^+,\varphi_{n,j}^- :  D_j \rightrightarrows D_n. \]
This is a cellular decomposition, corresponding to a partition of
$D_n$ into $2n+1$ open cells $D_n$, $S_j^+ = \varphi_{n,j}^+(D_j)$,
$S_j^- = \varphi_{n,j}^-(D_j)$. For any cell in this decomposition,
the incident cells are exactly those of strictly smaller dimension.

When introducing the operation $\overset{n}{\underset{\ell}{*}}$ with
$\ell=n-j$, it is seen that this corresponds to choosing a map
\[ D_n \to (D_n, S_j^+) \amalg_{D_j} (D_n, S_j^-),\]
satisfying a certain condition *, expressing the formulas I wrote down
yesterday for $s_1$ and $t_1$ of $u *_\ell v$ -- the formulas
translate into demanding that the restriction of the looked-for map of
$D_n$ to its boundary $S_{n-1}$ should be a given map (given at any
rate, for $\ell\ge2$, in terms of the operation $*_{\ell-1}$, which
explains the point I made that the $*_\ell$-structure is of order just
above the $*_{\ell-1}$-structure, namely (inductively) is of order
$\ell$ \ldots). That the extension of this map of $S_{n-1}$ to $D_n$
does indeed exist, comes from the fact that the amalgamated sum on the
right hand side is contractible for obvious reasons.

This gives a clue of what we should call ``standard'' amalgamated sums
of the cells $D_i$. The first idea that comes to mind is that we
should insist that the space considered should be contractible,
excluding amalgamated sums therefore such as
\[
\begin{tikzcd}[row sep=tiny]
  & \bullet \ar[dr] \\ \bullet \ar[ur] \ar[rr] & & \bullet
\end{tikzcd} \quad\text{or}\quad
\begin{tikzcd}
  \bullet \ar[r, bend left] \ar[r, bend right] & \bullet
\end{tikzcd}\]
which are circles. This formulation however has the inconvenience of
not being directly expressed in combinatorial terms. The following
formulation, which has the advantage of being of combinatorial nature,
is presumably equivalent to the former, and gives at any rate (I expect) a large
enough notion of ``standardness'' to yield for the corresponding
notion of \oo-category enough structure for whatever one will ever
need. In any case, it is\namedpspage{L.7prime}{7'} understood that the ``amalgamated
sum'' (rather, finite $\varinjlim$'s) we are considering are of the most
common type, when $X$ is the finite union of closed subsets $X_i$,
with given isomorphisms
\[ X \simeq  D_{n(i)},\]
the intersection of any two of these $X_i\cap X_j$ being a union of
closed cells both in $D_{n(i)}$ and in $D_{n(j)}$. (This implies in
fact that it is either a closed cell in both, or the union of two
closed cells of same dimension $m$ and hence isomorphic to $S_m$, a
case which will be ruled out anyhow by the triviality condition which
follows.\footnote{This is nonsense, as one sees already in the following picture where $f \cap f'' = S_0$: [picture of disk divided in three]}) The
triviality or ``standardness'' condition is now expressed by demanding
that the set of indices $I$ can be totally ordered, i.e., numbered in
such a way that we get $X$ by successively ``attaching'' cells
$D_{n(i)}$ to the space already constructed, $X(i-1)$, by a map from a
sub-cell of $D_{n(i)}$, $S_j^\xi \to X(i-1)$ ($\xi\in\{\pm1\}$), this
map of course inducing an isomorphism, more precisely \emph{the}
standard isomorphism, $\varphi_{n(i)}^\xi : S_j^\xi \simeq  D_j$ with
$S_j^\xi$ one of the two corresponding cells $S_j^+, S_j^-$ in some
$X_{i'}\simeq  D_{n(i')}$. The dual translation of this, in terms of
fiber products in a category $C$ endowed with objects $F_i$
($i\in\bN$) and maps $s_1^i,t_1^i$ between these, is clear: for a
given set of indices $I$ and map $i\mapsto n(i) : I \to \bN$, we
consider a subobject of $\prod_{i\in I} D_{n(i)}$, which can be
described by equality relations between iterated sources and targets
of various components of $u=(u_i)_{i\in I}$ in $P$, the structure of
the set of relations being such that $I$ can be numbers, from $1$ to
$N$ say, in such a way that we get in succession $N-1$ relations on
the $N$ components $u_i$ respectively ($2\le i\le N$), every relation
being of the type $f(u_i) = g(u_i')$, with $f$ and $g$ being iterated
source of target maps, and $i'<i$. (Whether source or target depending
in obvious way on the two signs $\xi,\xi'$.)

\newpage 

\presectionfill\ondate{21.2.}\namedpspage{L.8}{8}\par

% 8
\hangsection{Description of the universal \emph{primitive} structure.}%
\label{sec:8}%
Returning to the amalgamated sum $X = \bigcup_i X_i$, the cellular
decompositions of the components $X_i \simeq  D_{n(i)}$ define a
cellular decomposition of $X$, whose set of cells with incidence
relation forms a finite ordered set $K$, finite union of a family of
subsets $(K_i)_{i\in I}$, with given isomorphisms
\[ f_i : K_i \simeq  \cst \mathbf{J}_{n(i)} \quad (i\in I),\]
where for every index $n\in\bN$, $\cst \mathbf{J}_n$ denotes the ordered set of
the $2n+1$ cells $S_j^\xi$ ($0\le j\le n-1$, $\xi\in\{\pm1\}$), $D_n$
of the relevant cellular decomposition of $D_n$. We may without loss
of generality assume there is no inclusion relation between the $K_i$,
moreover the standardness condition described above readily translates
into a condition on this structure $K$, $(K_i)_{i\in I}$, $(f_i)_{i\in
  I}$, and implies that for $i,i'\in I$, $K_i\to K_{i'}$, is a
``closed'' subset in the two ordered sets $K_i,K_{i'}$ (namely
contains with any element $x$ the elements smaller than $x$), and
moreover isomorphic (for the induced order) to some $\cst \mathbf{J}_n$. Thus
the category $B_0$ can be viewed as the category of such ``standard
ordered sets'' (with the extra structure on these just said), and the
category $C_0$ can be defined most simply as the dual category
$B_0\op$. (NB\enspace the definition of morphisms in $B_0$ is clear I guess
\ldots) I believe the category $B_0$ is stable under amalgamated sums
$X \amalg_Z Y$, provided however we insist that the empty structure
$K$ is \emph{not} allowed -- otherwise we have to restrict to
amalgamated sums with $Z \ne \emptyset$. It seems finally more
convenient to exclude the empty structure in $B_0$, i.e.\ to exclude
the final element from $\bC_0$, for the benefit of being able to
state that $\bC_0$ (and all categories $\bC_n$\footnote{This is nonsense again for $n \leq 1$, see P.S. at the end of section 13. Even this P.S. is still inaccurate. Compare comments section 18.}) are stable under binary amalgamated sums,
and that the functors $\bC_n\to C$ (and ultimately $\bC_\oo\to
C$) we are interested in are just those commuting to arbitrary binary
amalgamated sums (without awkward reference to the objects $\boldsymbol F_i$
and the iterated source and target maps between them\ldots).

% 9
\hangsection[The main inductive step: just add coherence arrows! The \dots]%
{The main inductive step: just add coherence arrows! The abridged
  story of an \texorpdfstring{\textup(}{(}inescapable and
  irrelevant\texorpdfstring{\textup)}{)} ambiguity}\label{sec:9}%
The category $\bC_0$ being fairly well understood, it remains to
complete the construction by the inductive step, for passing from $\bC_n$
to $\bC_{n+1}$. The main properties I have in mind for this, for the
sequence of categories $\bC_n$ and their limit $\bC_\oo$, are the
following two.
\begin{enumerate}[label=(\Alph*)]
\item\label{it:8.A} For any $K\in\Ob(\bC_\oo)$ ($=\Ob(\bC_0)$), and any two
  arrows in $\bC_\oo$
  \[ f,g : K \rightrightarrows F_i,\]
  with $i\in\bN$, and such that either $i=0$, or the equalities
  \begin{equation}\label{eq:8.1}
    s_1^i\, f = s_1^i\, g , \quad
    t_1^i\, f = t_1^i\, g\tag{1}
  \end{equation}
  hold (case $i\ge1$), there exists $h : K \to F_{i+1}$ such that
  \begin{equation}\label{eq:8.2}
    s_1^{i+1}\, h = f, \quad
    t_1^{i+1}\, h = g.\tag{2}
  \end{equation}
\item\label{it:8.B}
  For any $n\in\bN$, the category $\bC_{n+1}$ is deduced from $\bC_n$
  by keeping the same objects, and just adding new arrows $h$ as
  in \ref{it:8.A}, with $f,g$ arrows in $\bC_n$.
\end{enumerate}
The\namedpspage{L.8prime}{8'} expression ``deduces from'' in \ref{it:8.B} means that we are
adding arrows $h: K\to F_i$ (each with preassigned source and target
in $\bC_n$), with as ``new axioms'' on the bunch of these uniquely
the two relations \eqref{eq:8.1}, \eqref{eq:8.2} of \ref{it:8.A}, the category $\bC_{n+1}$
being deduced from $\bC_n$ in an obvious way, as the solution of a
universal problem within the category of all categories where binary
amalgamated products exist, and ``maps'' between these being functors
which commute to those fibered products\footnote{Same mistake as the one noticed in the previous note. The fibered products exists in $\bC$ only, and \emph{these} should be preserved by the functors under consideration. Thus the ``universal problem'' has to b rephrased somewhat\dots}. In practical terms, the
arrows of $\bC_{n+1}$ are those deduced from the arrows in $\bC_n$ and
the ``new'' arrows $h$, by combining formal operations of
composing arrows by $v \circ u$, and taking (binary) amalgamated
products of arrows.\footnote{This has to be corrected -- amalgamated
  sums exist in $\bC$, only -- and \emph{those} should be
  respected.}

NB\enspace Of course the condition \eqref{eq:8.1} in \ref{it:8.A} is necessary for the
existence of an $h$ satisfying \eqref{eq:8.2}. That it is sufficient
too can be viewed as an extremely strong, ``universal'' version of
coherence conditions, concerning the various structures introduced on
an \oo-groupoid. Intuitively, it means that whenever we have \emph{two} ways
of associating to a finite family $(u_i)_{i\in I}$ of objects of an
\oo-groupoid, $u_i\in F_{n(i)}$, subjected to a standard set of
relations on the $u_i$'s, an element of some $F_n$, in terms of the
\oo-groupoid structure only, then we have automatically a ``homotopy''
between these built in in the very structure of the \oo-groupoid,
provided it makes at all sense to ask for one (namely provided
condition \eqref{eq:8.1} holds if $n \ge 1$). I have the feeling
moreover that conditions \ref{it:8.A} and \ref{it:8.B} (plus the
relation $\bC_\oo = \varinjlim \bC_n$) is all what will be ever
needed, when using the definition of the structure species, -- plus of
course the description of $\bC_0$, and the implicit fact that the
categories $\bC_n$ are stable under binary fiber products and the
inclusion functors commute to these.\footnote{Inaccurate; see above}
Of course, the category which really interests us is $\bC_\oo$, the
description of the intermediate $\bC_n$'s is merely technical --
the main point is that there should exist an increasing sequence
$(\bC_n)$ of subcategories of $\bC_\oo$, having the same objects (and
the ``same'' fiber-products), such that $\bC_\oo$ should be the
limit (i.e., every arrow in $\bC_\oo$ should belong to some $\bC_n$),
and such that the passage from $\bC_n$ to $\bC_{n+1}$
should satisfy \ref{it:8.B}. It is fairly obvious that these
conditions alone do by no means characterize $\bC_\oo$ up to
equivalence, and still less the sequence of its subcategories $\bC_n$.
The point I wish to make though, before pursuing with a proposal
of an explicit description, is (firstly) that \emph{this ambiguity is in the nature of things}. Roughly saying, two different mathematicians,
working independently on the conceptual problem I had in mind, assuming
they both wind up with some explicit definition, will almost certainly
get non-equivalent definitions -- namely with non-equivalent
categories of (set-valued, say) \oo-groupoids! And, secondly and as
importantly,\namedpspage{L.9}{9} that \emph{this ambiguity however is an
  irrelevant one}. To make this point a little clearer, I could say
that a third mathematician, informed of the work of both, will readily
think out a functor or rather a pair of functors, associating to any
structure of Mr.\ X one of Mr.\ Y and conversely, in such a way that
by composition of the two, we will associate to an $X$-structure ($T$
say) another $T'$, which will not be isomorphic to $T$ of course, but
endowed with a canonical \oo-equivalence (in the sense of Mr.\ X) $T
\underset{\oo}{\simeq } T'$, and the same on the Mr.\ Y side. Most
probably, a fourth mathematician, faced with the same situation as the
third, will get his own pair of functors to reconcile Mr.\ X and Mr.\
Y, which very probably won't be equivalent (I mean isomorphic) to the
previous one. Here however, a fifth mathematician, informed about this
new perplexity, will probably show that the two $Y$-structures $U$ and
$U'$, associated by his two colleagues to an $X$-structure $T$, while
not isomorphic alas, admit however a canonical \oo-equivalence between
$U$, and $U'$ (in the sense of the $Y$-theory). I could go on with a
sixth mathematician, confronted with the same perplexity as the
previous one, who winds up with another \oo-equivalence between $U$
and $U'$ (without being informed of the work of the fifth), and a
seventh reconciling them by discovering an \oo-equivalence between
these equivalences. The story of course is infinite, I better stop
with seven mathematicians, a fair number nowadays to allow themselves
getting involved with foundational matters \ldots
There should be a mathematical statement though resuming in finite
terms this infinite story, but in order to write it down I guess a
minimum amount of conceptual work, in the context of a given notion of
\oo-groupoids satisfying the desiderata \ref{it:8.A} and \ref{it:8.B}
should be done, and I am by no means sure I will go through this, not
in this letter anyhow.

% 10
\hangsection{Cutting down redundancies -- or: ``l'embarras du
  choix''.}\label{sec:10}%
Now in the long last the explicit description I promised of $\bC_{n+1}$
in terms of $\bC_n$. As a matter of fact, I have a handful
to propose! One choice, about the widest I would think of, is: for
every pair $(f,g)$ in $\bC_n$ satisfying condition \eqref{eq:8.1}
of \ref{it:8.A}, add one new arrow $h$. To avoid set-theoretic
difficulties though, we better first modify the definition of $\bC_0$
so that the set of its objects should be in the universe we are
working in, preferably even it be countable. Or else, and more
reasonably, we will pick one $h$ for every isomorphism class of
situations $(f,g)$ in $\bC_n$. Another restriction to avoid too
much redundancy -- this was the first definition actually that flipped
to my mind the day before yesterday -- is to add a \emph{new} $h$ only
when there is no ``old'' one, namely in $\bC_n$, serving the same
purpose. Then it came to my mind that there is a lot of redundancy
still, thus there would be already\namedpspage{L.9prime}{9'} infinitely many
operations standing for the single operation $v \overset{i}{\circ} u$
say, which could be viewed in effect in terms of an arbitrary
$n$-sequence ($n\ge2$) of ``composable'' $i$-objects $u_1=u, u_2=v,
u_3, \dots, u_n$, just ``forgetting'' $u_3, \dots, u_n$! The natural way to meet this ``objection'' would be
to restrict to pairs $(f,g)$ which cannot be factored non-trivially
through another objects $K'$ as
\[\begin{tikzcd}
  K \ar[r] & K' \ar[r, shift left, "f'"] \ar[r, shift right, swap,
  "g'"] & F_i \end{tikzcd}.\]
But even with such restrictions, there remain a lot of redundancies --
and this again seems to me in the nature of things, namely that there
is no really natural, ``most economic'' way for achieving condition
\ref{it:8.A}, by a stepwise construction meeting condition
\ref{it:8.B}. For instance, in $\bC_1$ already we will have not
merely the compositions $v \overset{i}{\circ} u$, but at the same time
simultaneous compositions
\begin{equation}
  \label{eq:10.star}
  u_n \circ u_{n-1} \circ \dots \circ u_1 \tag{*}
\end{equation}
for ``composable'' sequences of $i$-objects ($i\ge1$), without
reducing this (as is customary) to iteration of the binary composition
$v \circ u$.
Of course using the binary composition, and more generally iteration
of $n'$-ary compositions with $n'<n$ (when $n\ge3)$, we get an
impressive bunch of operations in the $n$ variables $u_1, \dots, u_n$,
serving the same purpose as \eqref{eq:10.star}. All these will be tied
up by homotopies in the next step $\bC_2$. We would like to think
of this set of homotopies in $\bC_2$ as a kind of ``transitive
system of isomorphisms'' (of associativity), now the transitivity
relations one is looking for will be replaced by homotopies again
between compositions of homotopies, which will enter in the picture
with $\bC_3$, etc. Here the infinite story is exemplified by the
more familiar situation of the two ways in which one could define a
``$\otimes$-composition with associativity'' in a category, starting
either in terms of a binary operation, \emph{or} with a bunch of $n$-ary
operations -- with, in both cases the associativity isomorphisms being
an essential part of the structure. Here again, while it is generally
(and quite validly) felt that the two points of view are equivalent;
and both have their advantages and their drawbacks, still it is not
true, I believe, that the two categories of algebraic structures
``category with associative $\otimes$-operation'', using one or the
other definition, are equivalent\footnote{I guess they are not equivalent, even when restricting to objects which are groupoids (i.e. so-called $\Gr$-categories).}. Here the story though of the relation between the two
notions is a \emph{finite} one, due to the fact that it is related to the
notion of $2$-categories or $2$-groupoids, instead of \oo-groupoids as
before\ldots

Thus I don't feel really like spending much energy in cutting down
redundancies, but prefer working with a notion of \oo-groupoid which
remains partly indeterminate, the main features being embodied in the
conditions \ref{it:8.A} and \ref{it:8.B} and in the description of
$\bC_0$, without other specification.\namedpspage{L.10}{10}

% 11
\hangsection[Returning to the topological model (the canonical
functor \dots]%
{Returning to the topological model \texorpdfstring{\textup(}{(}the
  canonical functor from spaces to
  ``\texorpdfstring{\oo}{oo}-groupoids''\texorpdfstring{\textup)}{)}.}%
\label{sec:11}%
One convenient way for constructing a category $\bC_\oo$ would be
to define for every $K,L\in\Ob(\bC_0) = \Ob(B_0)$ the set
$\Hom_\oo(K,L)$ as a subset of the set $\Hom(\abs L,\abs K)$ of
continuous maps between the geometric realizations of $L$ and $K$ in
terms of gluing together cells $D_i$, the composition of arrows in
$\bC_\oo$ being just composition of maps. This amounts to defining
$\bC_\oo$ as the dual of a category $\bB_\oo$ of topological
descriptions. It will be sufficient to define for every cell $D_n$ and
every subset $\Hom_\oo(D_n,\abs K)$ of $\Hom(D_n,\abs K)$, satisfying
the two conditions:
\begin{enumerate}[label=(\alph*)]
\item\label{it:11.a}
  stability by compositions $D_n \to \abs K \to \abs{K'}$, where $K\to
  K'$ is an ``allowable'' continuous map, namely subjected only to the
  condition that its restriction to any standard subcell $D_{n'}
  \subset \abs K$ is again ``allowable'', i.e., in $\Hom_\oo$.
\item\label{it:11.b}
  Any ``allowable'' map $S_n \to \abs K$ (i.e., whose restrictions to
  $S_n^+$ and $S_n^-$ are allowable) extends to an allowable map
  $D_{n+1}\to\abs K$.
\end{enumerate}

Condition \ref{it:11.a} merely ensures stability of allowable maps
under composition, and the fact that $\bB_\oo$ (endowed with the
allowable maps as morphisms) has the correct binary amalgamated sums,
whereas \ref{it:11.b} expresses condition \ref{it:11.a} on $\bC_\oo$.
These conditions are satisfied when we take as $\Hom_\oo$
subsets defined by tameness conditions (such as piecewise linear for
suitable piecewise linear structure on the $D_n$'s, or differentiable,
etc.). The condition \ref{it:11.b} however is of a subtler nature in
the topological interpretation and surely not met by such sweeping
tameness requirements only! Finally, the question as to whether we can
actually in this way describe an ``acceptable'' category $\bC_\oo$,
by defining sets $\Hom_\oo$, namely describing $\bC_\oo$ in terms
of $\bB_\oo$, seems rather subsidiary after all. We may think of
course of constructing stepwise $\bB_\oo$ via subcategories $\bB_n$,
by adding stepwise new arrows in order to meet condition
\ref{it:11.b}, thus paraphrasing condition \ref{it:8.B} for passage
from $\bC_n$ to $\bC_{n+1}$ -- but it is by no means clear that
when passing to the category $\bB_{n+1}$ by composing maps of $\bB_n$
and ``new'' ones, and using amalgamated sums too, there might not
be some undesirable extra relations in $\bB_{n+1}$, coming from the
topological interpretation of the arrows in $\bC_{n+1}$ as maps. To
say it differently, universal algebra furnishes us readily with an
acceptable sequence of categories $\bC_n$ and hence $\bC_\oo$,
and by the universal properties of the $\bC_n$ in terms of
$\bC_{n+1}$, we readily get (using arbitrary choices) a contravariant
functor $K \mapsto \abs K$ from $\bC_\oo$ to the category of
topological spaces (i.e., a co-\oo-groupoid in \cTop), but it is by no
means clear that this functor is \emph{faithful} -- and it doesn't really
matter after all!\namedpspage{L.10prime}{10'}

% 12
\hangsection{About replacing spaces by objects of a ``model
  category''.}\label{sec:12}%
I think I really better stop now, except for one last comment. The
construction of a co-\oo-groupoid $D_*$ in \cTop, giving rise to the
fundamental functor
\[ \cTop \longrightarrow (\text{\oo-groupoids}), \quad X \mapsto (\Hom(D_*, X))_i,
\]
generalizes, as I already alluded to earlier, to the case when \cTop{}
is replaced by an arbitrary ``model category'' $M$ in Quillen's sense. Here
however the choices occur not only stepwise for the primary,
secondary, ternary etc.\ structures, but already for the primitive
structures, namely by choice of objects $D_i$ ($i\in\bN$) in $M$, and
source and target maps $D_i \rightrightarrows D_{i+1}$. These choices
can be made inductively, by choosing first for $D_0$ the final object,
or more generally any object which is fibrant and trivial (over the
final objects), $D_{-1}$ being the initial object, and defining
further $S_0 = D_0 \amalg_{D_{-1}} D_0 = D_0 \amalg D_0$ with obvious
maps $\psi_0^+,\psi_0^- : D_0 \to S_0$, and then, if everything is
constructed up to $D_n$ and $S_n = (D_n,\varphi_{n-1}^+)
\amalg_{D_{n-1}} (D_n,\varphi_{n-1}^-)$, defining $D_{n+1}$ as any
fibrant and trivial\footnote{An object of $M$ is called fibrant (resp. trivial) if the map from it to the final object is fibrant (resp. a weak equivalence).} object together with a cofibrant map
\[ S_n \to D_{n+1}, \]
and $\varphi_n^+, \varphi_n^-$ as the compositions of the latter with
$\psi_n^+,\psi_n^-: D_n \rightrightarrows S_n$.
Using this and amalgamated sums in $M$, we get our functor
\[ \bB_0 = \bC_0\op \to M , \quad K\mapsto \abs K_M,\]
commuting with amalgamated sums, which we can extend stepwise through
the $\bC_n\op$'s to a functor $\bB_\oo = \bC_\oo\op \to M$,
provided we know that the objects $\abs K_M$ of $M$ ($K\in\Ob\bC_0$)
obtained by ``standard'' gluing of the $D_n$'s in $M$, are again
fibrant and trivial -- and I hope indeed that your axioms imply that,
via, say, that if $Z \to X$ and $Z\to Y$ are cofibrant and $X,Y,Z$ are
fibrant and trivial, then $X \amalg_Z Y$ is fibrant and trivial\footnote{This seems doubtful, unless all objects of $M$ are fibrant (which is true for the most familiar cases I was having in mind). But even assuming this, we still need a unicity statement (in a suitable sense) for the functors $\bB_0 \to $ thus obtained, in order to be sure that the corresponding functors $M \to \Hot$ are all canonical  isomorphic. These kind of questions may be viewed as closely related to the question of existence (and unicicity) of ``test functors'' for a given test category $A$ (here, the category of ``standard hemispheres'' $\Globe$) into a given asphericity or contractibility structure, as discussed in section 90 (without yet getting there any clear-cut handy existence and unicicity theorems as are to be hoped for). In my notes, the set-up of asphericity and contractibility structures, which has been worked out in the months following, has been gradually replacing Quillen's approach to homotopy models. (I'll have to come back in due course to the question of the relationship between the two approaches, which deserves to be understood.)}\ldots

Among the things to be checked is of course that when we localize the
category of \oo-groupoids with respect to morphisms which are ``weak
equivalences'' in a rather obvious sense (NB\enspace the definition of the
$\Pi_i$'s of an \oo-groupoid is practically trivial!), we get a
category equivalent to the usual homotopy category \Hot. Thus we get a
composed functor
\[ M \to (\text{\oo-groupoids}) \to \Hot,\]
as announced. I have some intuitive feeling of what this functor
stands for, at least when $M$ is say the category of semisimplicial
sheaves, or (more or less equivalently) of $n$-gerbes or \oo-gerbes\footnote{The word ``gerb'' here stands as a ``translation'' of the French word ``gerbe'', as used in Giraud's book on non-commutative cohomological algebra. With the terminology of next section, we should call it rather a ``stack of groupoids'' or ``$\Gr$-stack'' (with a specification by $n$ or $\infty$ if needed!)} on
a given topos: namely it should correspond to the operation of
``integration'' or ``sections'' for $n$-gerbes (more generally for
$n$-stacks) over a topos -- which is indeed \emph{the} basic operation
(embodying non-commutative cohomology objects of the topos) in
``non-commutative homological algebra''.

I guess that's about it for today. It's getting late and time to go to
bed! Good night.

\newpage 

% 13
\presectionfill\ondate{22.2.}1983\namedpspage{L.11}{11}\par

\hangsection[An urgent reflection on proper names: ``Stacks''
  and \dots]%
  {An urgent reflection on proper names: ``Stacks'' and
  ``coherators''.}\label{sec:13}%
It seems I can't help pursuing further the reflection I started with
this letter! First I would like to come back upon terminology. Maybe
to give the name of $n$-groupoids and \oo-groupoids to the objects I
was after is not proper, for two reasons: a) it conflicts with a
standing terminology, applying to structure species which are
frequently met and deserve names of their own, even if they turn out
to be too restrictive kind of objects for the use I am having in mind
-- so why not keep the terminology already in use, especially for
two-groupoids, which is pretty well suited after all; b) the structure
species I have in mind is not really a very well determined one, it
depends as we saw on choices, without any one choice looking
convincingly better than the others -- so it would be a mess to give
an unqualified name to such structure, depending on the choice of a
certain category $\bC_\oo = \bC$. I have been thinking of the
terminology \emph{$n$-stack} and \emph{\oo-stack} (stack = ``champ'' in French), a
name introduced in Giraud's book (he was restricting to champs =
1-champs), which over a topos reduced to a one-point space reduces in
his case to the usual notion of a category, i.e., $1$-category. Here
of course we are thinking of ``stacks of groupoids'' rather than
arbitrary stacks, which I would like to call (for arbitrary order
$n\in\bN$ or $n=\oo$) $n$-Gr-stack -- suggesting evident ties with the
notion of Gr-categories, we should say Gr-$1$-categories, of Mme Hoang
Xuan Sinh. One advantage of the name ``stack'' is that the use it had
so far spontaneously suggests the extension of these notions to the
corresponding notions over an arbitrary topos, which of course is what
I am after ultimately. Of course, when an ambiguity is possible, we
should speak of $n$-$\bC$-stacks -- the reference to $\bC$
should make superfluous the ``Gr'' specification. Thus $n$-$\bC$-stacks
are essentially the same as $n$-$\bC$-stacks over the
final topos, i.e., over a one point space. When both $\bC$ and
``Gr'' are understood in a given context, we will use the terminology
$n$-stack simply, or even ``stack'' when $n$ is fixed throughout. Thus
it will occur that in certain contexts ``stack'' will just mean a
usual groupoid, in others it will mean just a category, but when $n=2$
it will not mean a usual $2$-groupoid, but something more general,
defined in terms of $\bC$.

The categories $\bC=\bC_\oo$ described before merit a name too
-- I would like to call them ``\emph{coherators}'' (``\emph{coh\'ereurs}'' in
French). This name is meant to suggest that $\bC$ embodies a
hierarchy of coherence relations, more accurately of coherence
``homotopies''. When dealing with stacks, the term
\emph{$i$-homotopies} (rather than $i$-objects or $i$-arrows) for the
elements of the $i$\textsuperscript{th} component $F_i$ of a stack
seems to me the most suggestive -- they will of course be denoted
graphically by arrows\namedpspage{L.11prime}{11'} (such as $h: f\to g$ in the
formulation of \ref{it:8.A} yesterday). More specifically, I will call
coherator any category \emph{equivalent} to a category $\bC_\oo$ as
constructed before. Thus a coherator is stable under binary fiber
products,\footnote{false, see PS} moreover the $F_i$ are recovered up
to isomorphism as the indecomposable elements of $\bC$ with respect
to amalgamation. However, in a category $\bC_\oo$, the objects
$\boldsymbol F_i$ have non-trivial automorphisms -- namely the ``duality
involutions'' and their compositions (the group of automorphisms of
$\boldsymbol F_i$ should turn out to be canonically isomorphic to
$(\pm1)^i$),\footnote{This is false, as seen below} in other words by the mere
category structure of a coherator we will not be able to recover the
objects $\boldsymbol F_i$ in $\bC$ up to unique isomorphism. Therefore, in
the structure of a coherator should be included, too, the choice of
the basic indecomposable objects $\boldsymbol F_i$ (one in each isomorphism
class), and moreover the arrows $\cst s_1^i,\cst t_1^i : \boldsymbol F_i
\rightrightarrows \boldsymbol F_{i-1}$ for $i\ge1$ (a priori, only the
\emph{pair} $(\cst s_1^i,\cst t_1^i)$ can be described intrinsically
in terms of the category structure of $\bC$, once $\boldsymbol F_i$ and
$\boldsymbol F_{i-1}$ are chosen\ldots). But it now occurs to me that we
don't have to put in this extra structure after all -- while the
$\boldsymbol F_i$'s separately do have automorphisms, the system of objects
$(\boldsymbol F_i)_{i\in\bN}$ and of the maps $(\cst s_1^i)_{i\ge1}$ and $(\cst
t_1^i)_{i\ge1}$ has only the trivial automorphism (all this of course
is heuristics, I didn't really prove anything -- but the structure of
the full subcategory of a $\bC_\oo$ formed by the objects $\boldsymbol F_i$ seems
pretty obvious\ldots). To finish getting convinced that the
mere category structure of a coherator includes already all other
relevant structure, we should still describe a suitable intrinsic
filtration by subcategories $\bC_n$. We define the $\bC_n$
inductively, $\bC_0$ being the ``primitive structure'' (the arrows
are those deducible from the source and target arrows by composition
and fiber products), and $\bC_{n+1}$ being defined in terms of
$\bC_n$ as follows: add to $\Fl \bC_n$ all arrows of $\bC$ of
the type $h: K \to \boldsymbol F_i$ ($i\ge1$) such that $\cst s_1^i\, h$ and
$\cst t_1^i\,h$ are in $\bC_n$, and the arrows deduced from the
bunch obtained by composition and fibered products.\footnote{This description is dubious, as it may give categories $\bC_n$ which are larger that the ones we want.} In view of
these constructions, it would be an easy exercise to give an intrinsic
characterization of a coherator, as a category satisfying certain
internal properties.

I was a little rash right now when making assertions about the
structure of the group of automorphisms of $\boldsymbol F_i$ -- I forgot that
two days ago I pointed out to myself that even the basic operation
$\inv_i$ upon $\boldsymbol F_i$ need not even be involutions (nor
  even automorphisms)! However, I just checked that if in the inductive
construction of coherators $\bC_\oo$ given yesterday, we insist on
the most trivial irredundancy condition (namely that we don't add a
``new'' homotopy $h: f\to g$ when there is already an old one), then
any morphism $h: \boldsymbol F_i \to \boldsymbol F_i$ such that $\cst s\,f=\cst s$
and $\cst t\,f=\cst t$, is the identity -- and that\namedpspage{L.12}{12}
implies inductively that an automorphism of the system of $\boldsymbol F_i$'s
related by the source and target maps $\cst s_1^i,\cst t_1^i$ is the
identity. Thus it is correct after all, it seems, that the category
structure of a coherator implies all other structure relevant to
us\footnote{It seems dubious however that the mere category structure of $\bC$ will allow us to recover the ``primitive'' subcategory $\bC_0$, and it looks safer to add the latter as an extra structure to $\bC$}.

I do believe that the description given so far of what I mean by a
coherator, namely something acting like a kind of pattern in order to
define a corresponding notion of ``stacks'' (which in turn should be
the basic coefficient objects in non-commutative homological algebra,
as well as a convenient description of homotopy types) embodies some
of the essential features of the theory still in embryo that wants to
be developed. It is quite possible of course that some features are
lacking still, for instance that some extra conditions have to be
imposed upon $\bC$, possibly of a very different nature from mere
irredundancy conditions (which, I feel, are kind of irrelevant in this
set-up). Only by pushing ahead and working out at least in outline the
main aspects of the formalism of stacks, will it become clear whether
or not extra conditions on $\bC$ are needed. I would like at least
to make a commented list of these main aspects, and possibly do some
heuristic pondering on some of these in the stride, or afterwards. For
today it seems a little late though -- I have been pretty busy with
non-mathematical work most part of the day, and the next two days I'll
be busy at the university. Thus I guess I'll send off this letter
tomorrow, and send you later an elaboration (presumably much in the
style of this unending letter) if you are interested. In any case I
would appreciate any comments you make -- that's why I have been
writing you after all! I will probably send copies to Ronnie Brown,
Luc Illusie and Jean Giraud, in case they should be interested (I
guess at least Ronnie Brown is). Maybe the theory is going to take off
after all, in the long last!

\bigskip

Very cordially yours

\bigskip

\hfill Alexander

\bigskip

PS (25.2.)\enspace I noticed a rather silly mistake in the notes of two days
ago, when stating that the categories $\bC_n$ admit fiber products:
what is true is that the category $\bC_0$ has fiber products (by
construction practically), and that these are fiber products also in
the categories $\bC_n$ (by construction equally), i.e., that the
inclusion functors $\bC_0 \to \bC_n$ commute to fiber
products. Stacks in a category $C$ correspond to functors $\bC_\oo\to
C$ whose \emph{restriction to} $\bC_0$ commutes with fiber
products. I carried the mistake along in the yesterday notes -- it
doesn't really change anything substantially. I will have to come back
anyhow upon the basic notion of a coherator\ldots











\chapter*{Appendix: Three letters to Larry Breen}
\addcontentsline{toc}{chapter}{Appendix}
\label{ch:AppchI}

In this appendix, I am including three letters to Larry Breen\scrcomment{Lawrence Breen}, dated 5.2, 
17.2 and 17-19.7.1975. These letters are written in French, and the first 
two are reproduced textually, whereas the third appears here in English translation.
I am grateful to Ronnie Brown who took the trouble last year to make such a translation 
(from a hardly legible copy of the handwritten letter to Larry Breen) with the assistance of Larry Breen himself and J. L. Loday. I am now using his translation rather than the original letter, which no printer could possibly decipher!
Also, for the second letter I am using a typed copy made in 1945 by Larry Breen (who presumably had difficulties too deciphering the handwriting). Thanks are due to him for his interest and patience (with someone like me, very unknowledgeable in standard homotopy techniques), which appeared in the letters I got from him in response and his verbal explanations on related matters, as well as in the trouble he took in retyping the letter and sending me copies of my own letters to him, and allowing me to reproduce them in this volume of Pursuing stacks. My only present contribution to this set of letters is adding a few comments (in the notes), adding subtitles (with numbers 1 to 18), and correcting some inaccuracies in the English translation (due mainly to my handwriting\dots). 
Also, I skipped the beginning of the first letter, which doesn't seem of general relevance.

The first two letters are an attempt to explain to Larry Breen (who has a wide background in algebraic geometry and homological and homotopical algebra) some of the main points of the programme I had in mind around the notions of $n$-categories and $n$-stacks (which is what I am supposed to be pursuing now in my present work ``Pursuing stacks''). They were written under the impetus of the new intuition (new to me at any rate) which then had just appeared to me, namely that (non strict) $n$-groupoids should model (in a suitable sense) $n$-truncated homotopy types. The third letter, written in answer to a number of questions in Larry Breen's response to the first two, is of a wider scope. A large part of the letter outlines (very sketchily) some main points of a duality program (including a cohomological formulation of ``geometric'' local and global class field theory), which emerged by the end of the fifties and appears here for the first time in print. The later part of the letter gives also some hints about the need of a framework of ``tame topology'' suitable for writing up a ``dévissage theory'' of stratified spaces, and for working with étale tubular neighbourhoods, for the common purpose of coming to grips with a suitable notion of ``finite homotopy type'' of a ``tame'' topological space or of a scheme say, in terms of the inverted system of ``indexed homotopy types'' corresponding to all equi-singular stratifications.

\bigskip

\par\hfill Villecun \ondate{5.2.}1975\par

Cher Breen

\dots

\setcounter{section}{0}

%App1
\hangsection{Examples de $2$-catégories de Picard.}\label{sec:app1}%
\dots Pour tout le reste de ta lettre, elle mériterait un lecteur plus averti, aussi, pour qu'elle ne soit pas entièrement perdue au monde, je vais l'envoyer à Illusie ! J'ai néanmoins constaté, avec intérêt, ton intérêt â demi refoulé pour des $2$-catégories de Picard, $n$-catégories et autre faune de ce genre, et ton espoir que je te prouverai peut-être que ces animaux sont tout à fait indispensables pour faire des maths sérieuses dans telle circonstance. J'ai bien peur que cet espoir ne soit dé\c{c}u, je crois que jusqu'à maintenant on a toujours pu d'en tirer en éludant de tels objets et l'engrenage dans lequel ils pourraient nous entraîner. Est-ce nécessairement une raison pour continuer à les éluder ? Les situations où on a l'impression ``d'éluder'' en effet me semblent en tous cas devenir toujours plus nombreuses - et si on s'abstenait de tirer une situation complexe et chargée de mystère au clair, chaque fois qu'on ne serait pas \emph{forcé} de le faire pour des raisons techniques provenant de la math déjà faite, - il y aurait sans doute beaucoup de parties des maths aujourd'hui reputées ``sérieusses'' qui n'auraient jamais été developpées (Il n'est pas dit non plus que le mode s'en trouverait plus mal...). Ton commentaire (que j'ai également entendu chez Deligne) que la classification d'objets géométriques relativement merdiques se réduit finalement à des invariants cohomologiques essentiellement ``bien connus'' et relativement simples n'est pas non plus convainquant; n'est ce pas négliger la différence entre la \emph{compréhension d'un objet géométrique}, et la détermination de sa ``classe à isomorphisme (ou équivalence) près'' ?

Tu me demandes des exemples ``convainquants'' de $2$-catégories de Picard. Voici quelques exemples, en vrac (je ne sais s'ils seraient convainquants !):

\begin{enumerate}
 \item[1)]\label{it:app1.1} Si $L$ est un lien\footnote{For the notion of a ``lien'' (or ``tie''), which is one of the main ingredients of the non-commutative cohomology panoply of Giraud's theory, I refer to his books (Springer, Grundlehren 179, 1971). A \emph{Picard category} is a groupoid endowed with an operation $\otimes$ together with associativity, unity and commutativity data for this operation, which make it resemble to a commutative group. A \emph{``Champ de Picard''} (or ``Picard stack'') is defined accordingly, by relativizing over an arbitrary space or topos (replacing the groupoid by a stack of groupoids over this topos). The necessary ``general nonsense'' on these notions is developed rather carefully in an exposé of Deligne in SGA 4 (SGA 4 XVIII 1.4). In this letter to Larry Breen, I am asuming ``known'' the notion of an $n$-stack (for $n = 3$ at any rate), and the corresponding notion of (strict) \emph{Picard $n$-stack}, which should be describable (as was explained in Deligne's notes in the case $n = 1$) by an $n$-truncated chain complex in the category of abelian sheaves on $X$ (viewed mainly as an object of the relevant derived category). The ``strictness'' condition on usual Picard stacks refers to the restriction that the commutativity isomorphism within an object $L \otimes L'$, when $L = L'$, should reduce to the identity. It is assumed (without further explanation) that the condition carries over in a natural way to Picard $n$-stacks, in such a a way as to allow an interpretation of these by truncated objects in a suitable derived category, as hinted above.} de centre $Z$ sur le topos $X$, les \emph{gerbes liées par $Z$} forment une $2$-catégorie de Picard stricte, représentée par le complexe $\RGamma_X(Z)$ tronqué en degré $2$, dont les objets de cohomologie non triviaux sont les $\mathrm H^i(X, Z)$, $0 \leq i \leq 2$. Les gerbes liées par $L$ forment un \emph{pseudo-$2$-torseur} sous le gerbe précédente, qui est un $2$-torseur (i.e. non vide) si et seule si une certaine obstruction dans $\mathrm H^3(X, Z)$ est nulle. Pour comprendre cette classe de notre point de vue, il y a lieu de passer aux $2$-champs correspondants: le $2$-champs de Picard strict des $Z$-gerbes sur des objets variables de $X$, et le $2$-champ des $L$-gerbes sur des objets variables. Ce dernier est bel et bien un $2$-torseur sous le champ précédent, or la classification de ces $2$-torseurs (à $2$-équivalence près) se fait par le $\mathrm H^3(X, Z)$, (tout comme les $Z$-$L$-gerbes peuvent être interprétées comme des torseurs sous la $Z$-$L$-champ de Picard strict des $Z$-torseurs, et sont classifiées  par le $\mathrm H^2(X, Z)$). On voit déjà, bien sûr, poindre ici l'oreille de la $3$-catégorie de Picard stricte des $2$-gerbes liées par $Z$, ou (de fa\c{c}on équivalente) des $2$-torseurs sous le $2$-champ de Picard strict des $Z$-$L$-gerbes; cette $3$-catégorie de Picard stricte étant décrite par $\RGamma_X(Z)$ tronqué en dimension $3$, ayant comme invariants de cohomologie non triviaux les $\mathrm H^i(X, Z)$ ($0 \leq i \leq 3$). Quant au $3$-champ de Picard correspondant, il est décrit par une résolution injective de $Z$ tronqué en degré $3$, alors que le $2$-champ de Picard précédent se décrivait en tronquant en degré $2$.
    
\item[2)]\label{it:app1.2} Si $M$ et $N$ sont deux faisceaux abéliens sur $X$, les \emph{champs de Picard} (N.B. $1$-champs !) \emph{d'invariants $M$ et $N$} forment eux-même une $2$-catégorie de Picard stricte, représentée sans doute par le complexe $\RHom(X(M), N)$\footnote{When $M$ is any abelian sheaf on a topos, the ``MacLane resolution'' $X(M)$ is a certain canonical left resolution of $M$ by sheaves of $\mathbf{Z}$-modules which are ``free'', and more specifically, which are finite direct sums of sheaves of the type $\mathbf{Z}^{(T)}$, where $T$ is any sheaf of the type $M^n$ (finite product of copies of $M$). This canonical construction was introduced by MacLane (for abelian groups), and gained new popularity in the French school of algebraic geometry and homological algebra in the late sixties, because it gives a very handy way to relate the $\Ext^i(M, N)$ invariants (when $N$ is another abelian sheaf on $X$) to the ``spacial'' cohomology of $M$ (i.e. of the induced topos $X_{/M}$) with coefficients in $N$.} tronqué en degré $2$, dont les invariants de cohomologie non triviaux sont donc ``le drôle de $\Ext^2$'' de ma lettre à Deligne, et les honnêtes $\Ext^i(M, N)$ ($0 \leq i \leq 2$). Les champs de Picard stricts forment une sous-$2$-catégorie de Picard pleine, représentée par $\RHom(M, N)$ tronqué en degré $2$, d'invariants les $\Ext^i(M; N)$ ($0 \leq i \leq 2$). Bien sûr, $\Ext^2$ donne les $0$-objets à équivalence près, $\Ext^1$ les automorphismes à isomorphisme près de l'objet nul, $\Ext^0$ les automorphismes de l'automorphisme identique audit\dots Je n'ai pas réfléchi à une bonne interprétation géométrique de la $n$-catégorie de Picard associée à $\RHom(M, N)$ tronqué en degré $n$, et encore moins bien sûr pour $\RHom(X(M), N)$, mais sans doute il faut regarder dans la direction des $n$-champs de Picard. 
    
\item[3)]\label{it:app1.3} Soit $G$ un Groupe sur $X$, opérant sur un faisceau abélien $N$. Les \emph{champs en $\Gr$-catégories sur $X$ liés par $(G, N)$} forment une $2$-catégorie de Picard, dont les invariants sont $\mathrm H^3(\B_G mod X, N)$, $\mathrm H^2(\B_G mod X, N)$ et $Z^1(G, N)$ (groupe des $1$-cocycles de $G$ à coefficients dans $N$) - je te laisse le soin de deviner quel est le complexe qui le décrit ! J'ai écrit il y a quelques mois à Deligne\scrcomment{cf.\ \textcite{LGD7874}} à ce sujet, et l'ai prié de t'envoyer une copie de la lettre.
    
\item[4)]\label{it:app1.4} Soit $X$ un topos localement annelé, on peut considérer les \emph{Algèbres d'Azumaya sur $X$} (i.e. les Algèbres localement isomorphes à une algèbre de matrices d'ordre $n$, $n \geq 1$) comme les objets d'une $2$-catégorie de Picard, où la catégorie $\bHom(A, B)$, pour $A$ et $B$ des Algèbres d'Azumaya, est la catégorie des ``trivialisations'' de $A^{\circ} \otimes B$, i.e. des couples $(E, \emptyset)$, $E$ un Module localement libre et $\emptyset$ un isomorphisme $\bEnd(E) \simeq A^{\circ}\otimes B$. Il faut travailler un peu pour définir les accouplements $\bHom(A, B) \times \bHom(B, C) \to \bHom(A, C)$; l'opération $\otimes$ dans la $2$-catégorie de Picard à construire est bien sûr le produit tensoriel d'Algèbres, et l'opération ``puissance $-1$'' est le passage à l'algèbre opposée. On vérifie qu'en associant à toute Algèbre d'Azumaya la $1$-gerbe de ses trivialisations, on trouve un $2$-$\otimes$-foncteur de la $2$-catégorie de Picard (dite ``de Brauer'') dans celle des $1$-gerbes liées par $\mathbf{G}_m$, qui est $2$-fidèle. Les invariants de la première sont donc les groupes $\mathrm H^2(X, \mathbf{G}_m)_{\Br}$, $\mathrm H^1(X, \mathbf{G}_m)$ et $\mathrm H^0(X, \mathbf{G}_m)$, où dans le premier terme l'indice $\Br$ désigne le sous-groupe du $\mathrm H^2$ formé des classes de cohomologie provenant d'Algèbres d'Azumaya. On aurait envie de parler du $2$-champ de Picard des Algèbres d'Azumaya sur des objets variables de $X$, mais c'est bien une $2$-catégorie de Picard fibrée sur $X$, mais pas tout à fait un $2$-champ (whatever that means), san doute - la condition de $2$-recollement (whatever that means) ne doit pas être satisfaite - sinon il n'y aurait pas d'indice $\Br$ au $\mathrm H^2$...
\end{enumerate}

\hangsection{Théorème de Lefschetz (faible) en termes de Champs.}\label{sec:app2}%
La considération des $n$-catégories de Picard strictes (qui s'imposent à nous pas à pas dans un contexte essentiellement ``commutatif'') me semblent la clef du passage de l'algèbre homologique ordinaire (``commutative''), en termes de complexes, à une algèbre homologique non commutative, du fait qu'elles donnent une interprétation géométrique correcte des ``complexes tronqués à l'ordre $n$'' (en tant qui objets de catégories dérivées), donc, essentiellement (par passage à la limite sur $n$) des complexes tout courts. L'idée naïve qui se présente est alors que les ``complexes non commutatifs'' (qui seraient les objets-fantômes d'une algèbre homologique non commutative) sont peut-être ce qui reste des $n$-catégories de Picard (strictes) quand on oublie leur caractère additif, i.e. leur structure de Picard - c'est à dire qu'on ne retient que la $n$-catégorie ! (Quand on se place sur un topos $X$, on s'intéresse donc aux $n$-champs sur $X$...) A vrai dire, cette idée est venue d'abord d'une autre direction, quand il s'est agi en géométrie algébrique de démontrer des \emph{théorèmes de Lefschetz} à coefficients discrets en cohomologie étale, dans le cas d'une variété projective disons et de toute section hyperplane, ou d'une variété quasi-projective et de presque toute section hyperplane (pour ne mentionner que le cas global le plus simple), sous les hypothèses de profondeur cohomologique ``le plus naturelles'' (en fait, essentiellement des conditions nécessaires et suffisantes de validité du dit théorème). Dans le cas commutatif, les techniques de dualité nous suggèrent très clairement quels sont les meilleurs énoncés possibles, cf. l'exposé de Mme Raynaud dans SGA 2\scrcomment{\textcite{SGA2}}. Mais ces techniques ne valent qu'en se restreignant à des coefficients premiers aux caractéristiques, alors que des démonstrations directes plus géométriques (développées dans SGA 2 avant le développement du formalisme de la cohomologie étale) donnaient des résultats très voisins pour le $\mathrm H^0$ et le $\mathrm H^1$ (ou le $\pi_0$ et le $\pi_1$, si on préfère) sans telles restrictions, du moins dans le cas propre (i.e. projectif, au lien de quasi-projectif). En fait, ce sont les ``résultats les meilleurs possibles'' eux-mêmes, énoncés comme conjectures dans SGA 2 dans l'exposé cité de Mme Raynaud, qui sont démontres ultérieurement par elle dans sa thèse\scrcomment{\textcite{Raynaud1975}}. Ce qui est remarquable de notre point de vue, c'est que les énoncés les plus forts se présentent le plus naturellement sous forme d'énoncés sur des \emph{1-champs} sur le site étale de la variété algébrique considérée - la notion de ``profondeur $\geq i$ (pour $i = 1, 2, 3$) s'énon\c{c}ant aussi le plus naturellement en termes de champs. Non seulement cela, mais alors même qu'on voudrait ignorer la notion technique de champ et travailler exclusivement en termes de $\mathrm H^0$ et $\mathrm H^1$ en utilisant à bloc le formalisme cohomologique non commutatif de Giraud, pour démontrer disons un théorème de bijectivité $\pi_1(Y) \to \pi_1(X)$ (ce qui est le résultat le plus profond établi dans la thèse de Mme Raynaud), il semble bien qu'on n'y arrive pas, faute à ce formalisme d'avoir la souplesse nécessaire. En fait, il faut utiliser comme ingrédients techniques, de fa\c{c}on essentielle, les trois théorèmes suivantes directement pour les $1$-champs ``de torsion'' (i.e. où les faisceaux en groupes d'automorphismes sont de ind-torsion): a) théorème de changement de base pour une morphisme propre, b) théorème de changement de base par un morphisme lisse c) théorème de ``propreté cohomologique générique'' pour un morphisme de type fini $f: X \to S$, $S$ intègre (disant que l'on peut trouver dans $S$ un ouvert $U \neq \emptyset$ tel que pour \emph{tout} changement de base $S' \to S$ se factorisant par $u$, la formule de changement de base est vraie). (Pour b) et c), il faut faire des hypothèses que les faisceaux d'automorphismes sont premiers aux charactéristiques, et dans c) ne servent que dans la version ``générique'' du théorème de Lefschetz). C'est avec en vue de telles applications que Giraud a pris la peine dans son bouquin (si je ne me trompe) de démontrer a), b) (et c) ?) dans le contexte des 1-champs et de leurs images directes et inverses. Mais du même coup il dévient clair que le contexte ``naturel'' des théorèmes de changement de base en cohomologie étale, des théorèmes du type de Lefschetz (dits ``faibles'') sur les ``sections hyperplanes'', tout comme de la notion de profondeur qui y joue un rôle crucial, doit être celui des $n$-champs. Et que le développement hypothétique de ce contexte ne risque pas de se réduire à une jonglerie purement formelle et absolument bordélique avec du ``general nonsense'', mais qu'on se trouvera aussitôt confronté à des tests ``d'utilisabilité'' aussi sérieux que la démonstration des théorèmes de changement de base et ceux du type de Lefschetz (qui même dans le contexte commutatif ne sont pas piqués de vers...). [] pour variantes analytiques complexes etc.

Je ne sais si ces commentaires te ``passent par dessus la tête'' à ton tour, ni si elles te donnent l'impression qu'il aurait peut-être des choses intéressantes à tirer au clair. Si cela t'intéresse, je pourrais expliciter sous forme un peu plus systématique quelques ingrédients d'une hypothétique algèbre homologique non commutative et les liens de celle-ci à l'algèbre homologique commutative. Plus mystérieux pour moi (et pour cause, vu mon ignorance en homotopie) seraient les relations entre celle-là et l'algèbre homotopique, i.e. les structures semi-simpliciales, et je n'ai que des commentaires assez vagues à faire en ce sens (*). Par ailleurs, je te rappelle que même l'algèbre homologique commutative n'est pas, il s'en faut, dans un état satisfaisant, pour autant que je sache, vu qu'on ne sait\footnote{Reflecting on the ``right'' version of the provisional Verdier notion of a triangulated category (which was supposed to describe adequately the relevant internal structure of the derived categories of abelian categories) is part of my present program for the notes on Pursuing stacks, and will be the main task in one of the chapters of volume two. For some indications along these lines, see also section 69 (sketching the basic notion of a ``\emph{derivator}'').} toujours pas quelle est la ``bonne'' notion de catégorie triangulée. Or il me semble bien clair que ce n'est pas une question purement académique - même si on a pu se passer de le savoir jusqu'ici (en se bornant comme Monsieur Jourdain à ``faire de la prose sans le savoir'' - en travaillant sur des catégories de complexes, éventuellement filtrés, sans trop se demander quelles structures il y a sur ces catégories...).

Bien cordialement à toi

\hangsection{Les $n$-groupoïdes comme types d'homotopie tronqués.}\label{sec:app3}%
(*) P.S. Réflexion faite, j'ai quand même envie de te mettre un peu en appétit, en faisant ces ``quelques commentaires assez vagues''. Il s'agit du yoga qu'une (petite) $n$-catégorie ou groupoïdes (à $n$-équivalence près) ``est essentiellement la même chose'' qu'une ensemble semi-simplicial pris à homotopie près et où on néglige les $\pi_i$ pour $n + 1 \geq i$ (où, si tu préfères, ``où on a tué les groupes d'homotopie en dimension $\geq n + 1$). Voici des éléments heuristiques pour ce yoga. Si $K_\bullet$ est un ensemble simplicial (il peut être prudent de le prendre de Kan) on lui associe une $n$-catégorie $C_n(K_\bullet)$, dont les $0$-objets sont les $0$-simplexes, les $1$-objets sont les chemins (ou homotopies) entre $0$-simplexes, les $2$-objets sont les homotopies entre chemins (à extrémités fixées) etc. Pour les $n$-objets, cependant, on ne prend pas les homotopies entre homotopies de fourbis, mais classes d'équivalence de homotopies (modulo la relation d'homotopie) entre homotopies. La composition des $i$-objets ($i \geq 1$) se définit de fa\c{c}on évidente, on notera qu'elle n'est pas strictement associative, mais associative modulo homotopie. Donc la $n$-catégorie qu'on obtient n'est pas ``stricte'' - et on prévoit pas mal d'emmerdement pour définir de fa\c{c}on raisonnable une $n$-catégorie pas stricte (dans la description des compatibilités pour les ``données d'associativité). La mise sur pied du yoga qui suit pourrait constituer un fil d'Ariadne pour la définition en forme des $n$-catégories (pas strictes), les $n$-foncteurs entre elles (pas non plus stricts, et pour cause), les $n$-équivalences etc, au même titre que le yoga initial ``une $n$-catégorie est une catégorie ou les $\Hom$ et leurs accouplements de composition sont des $(n-1)$-catégories et des accouplements entre telles''. Cette $n$-catégorie $C_n(K_{\bullet})$ dépend fonctoriellement de $K_{\bullet}$, tout morphisme simplicial $K_{\bullet} \to K'_{\bullet}$ définit un $n$-foncteur $C_n(K_{\bullet} \to C_n(K'_{\bullet}))$ ; en fait, cela doit en dépendre même $n$-fonctoriellement, vu qu'on voit (en s'inspirant de ce qui précède et l'application à des ensembles semi-simpliciaux de la forme $\Hom_{\bullet}(K_{\bullet}, K'_{\bullet})$) que les ensembles semi-simpliciaux forment eux-mêmes les $0$-objets d'une $n$-catégorie, quel que soit $n$\dots

En fait, $C_n(K_{\bullet})$ est un $n$-groupoïde, i.e. une $n$-catégorie où toute $i$-flèche ($1 \leq i \leq n$) (= $i$-objet) est une ``équivalence'' i.e. admet un quasi-inverse (donc un inverse si la $n$-catégorie est ``réduite''). Si $C$ est une telle $n$-catégorie i.e. un $n$-groupoïde, et $X$ un $0$-objet de $C$, il s'impose de désigner par $\pi_i(C, x)$ ($0 \leq i \leq n$) successivement : l'ensemble des classes de $0$-objets à équivalence près de $1$-objets (ou $1$-flèches) $x \to x$ (c'est un groupe, pas nécessairement commutatif), l'ensemble des classes modulo équivalence des $2$-flèches $1_x \to 1_x$, où $1_x$ est la $1$-flèche identique de $x$ (c'est un groupe commutatif $\pi_2(C, x)$, ainsi que les groupes qui vont suivre), l'ensemble des classes modulo équivalence de $3$-flèches $1_{1_x} \to 1_{1_x}$, etc. Ces groupes forment, comme de juste, des ``systèmes locaux'' sur l'ensemble des $0$-objets de $C$, et modulo le grain de sel habituel, les $\pi_i(C, x)$ ne dépendent que de la ``composante connexe'' du $0$-objet $x$ i.e. de sa classe modulo équivalence de $0$-objets. Ceci dit, si $C$ est de la forme $C_n(K_{\bullet})$, il résulte pratiquement des définitions que l'on a des isomorphismes canoniques $\pi_i(K_{\bullet}, x) \simeq \pi_i(C_n(K_{\bullet}))$ pour $0 \leq i \leq n$, qui pour $x$ variable peuvent s'interpréter comme des isomorphismes de systèmes locaux. Il s'ensuit que pour une application semi-simplicial $f: K_{\bullet} \to K'_{\bullet}$, le $n$-foncteur correspondant $C_n(K_{\bullet}) \to C_n(K'_{\bullet})$ est une $n$-équivalence si et seule si $f$ induit un isomorphisme sur les $\pi_0$ et sur les $\pi_i$ en tout point ($1 \leq i \leq n$). On serait plus heureux de pouvoir dire à la place ``et de plus un homomorphisme surjectif pour $i = n + 1$, car c'est, il me semble, cela qu'il faudrait pour espérer pouvoir conclure que la catégorie localisée de la catégorie des ensembles semi-simpliciaux, obtenue en inversant les flèches ``qui induisent des isomorphismes sur les $\pi_i$ pour $0 \leq i \leq n$ (ou encore, ``en négligeant'' les ensembles semi-simpliciaux $n$-connexes), est équivalente à la catégorie localisée de la catégorie des $n$-catégories, où on rend inversibles les $n$-équivalences ? Quoi qu'il en soit, ces petites bavures devraient disparaître lorsqu'on ``stabilise'' en faisant augmenter $n$. A ce propos, on voit que le foncteur ``troncature en dimension $n$'' de la théorie homotopique (consistant à tuer les groupes d'homotopie à partir de la dimension $n + 1$) s'interprète dans la langage des $n$-catégories par l'opération faisant passer d'une $N$-catégorie ($N > n$) à une $n$-catégorie, en conservant tels quels les $i$-objets ($0 \leq i \leq n-1$) et leur composition ($1 \leq i \leq n-1$), et en rempla\c{c}ant les $n$-objets par les classes de $n$-objets ``à équivalence près'', avec la composition obtenue par passage au quotient. De même, le foncteur d'inclusion évident en théorie homotopique, consistant à regarder un ensemble semi-simplicial ``où on a négligé les $\pi_i$ pour $i \geq n + 1$'' comme un ensemble semi-simplicial (dans la catégorie homotopique) qui se trouve avoir des $\pi_i$ nuls pour $i \geq n + 1$, se traduit par le foncteur allant des $n$-catégories vers les $N$-catégories, obtenue en ajoutant à une $n$-catégorie des $i$-flèches ($n + 1 \leq i \leq N$) identiques exclusivement. (Ainsi, un ensemble est regardé comme une catégorie ``discrète'', une catégorie comme une $2$-catégorie où les $\bHom(A, B)$, $A$ et $B$ des $0$-objets, sont des catégories discrètes, etc\dots).

Bien entendu, rien n'empêche de considérer aussi la notion de $\infty$-catégorie, à laquelle celle de $n$-catégorie est comme la notion d'ensemble semi-simplicial tronqué à celle d'ensemble semi-simplicial. Sauf erreur, la localisée de la catégorie des $\infty$-catégories, pour les flèches de $\infty$-équivalence, est équivalente à ``la catégorie homotopique'', localisée de la catégorie des ensembles semi-simpliciaux, ou du moins une sorte de complétée de celle-là. Dans cette optique, le tapis consistant à interpréter une $\infty$-catégorie de Picard stricte (i.e. quelque chose qui ressemble à un groupe abélien de la catégorie des $\infty$-catégories) comme donnée (à $\infty$.équivalence près) par un complexe de chaînes regardé comme un objet d'une catégorie dérivée, est à relier au tapis de Dold-Puppe, interprétant ces derniers comme des groupes abéliens semi-simpliciaux.

Pour se donner confiance dans ce yoga général, on peut essayer d'interpréter en termes de $n$-catégories ou $\infty$-catégories des constructions familières en homotopie. Ainsi, l'espace des lacets $\Omega (K_\bullet, x)$ correspond manifestement à la $(n-1)$-catégorie $\bHom(x, x)$ formée des $i$-flèches de $C$ $(1 \leq i \leq n)$ dont la 0-origine et la 0-extrémité sont $x$, réindexées en les appelant $(i-1)$-flèches. Je n'aper\c{c}ois pas à vue de nez un joli candidat pour la suspension en termes de $n$-catégories. Par contre le $\Hom_\bullet (K_\bullet, K_\bullet')$ doit correspondre au $\bHom(C, C')$, qui est une $n$-catégorie quand $C$, $C'$ en sont. La ``fibre homotopique'' d'une application semi-simpliciale $f: K_\bullet \to K_\bullet'$ (transformée d'abord, pour les besoins de la cause, en une fibration de Serre par le procédé bien connu de Serre-Cartan) correspond sans doute à l'opération bien familière de produit $(n+1)$-fibré (du moins les cas $n = 0, 1$ sont bien familiers !) $C \times_{C'} C''$ pour des $n$-foncteurs $c \to C'$ et $C'' \to C'$, dans le cas où $C''$ est la $n$-catégorie ponctuelle, donc la donnée de $C'' \to C'$ correspond à la donnée d'un 0-objet de $C'$. Les espaces $K(\pi, n)$ ont une interprétation évidente comme $n$-gerbes liées par $\pi$. Enfin, on voit aussi poindre l'analogue du dévissage de Postnikoff d'un ensemble semi-simplicial - mais la fa\c{c}on dont je l'entrevois (vue ma prédilection pour les topos) passe par la notion de topos classifiant d'un $n$-groupoïde (généralisant de fa\c{c}on évidente le topos classifiant d'un groupe). En termes de cette notion, on peut, il me semble, interpréter un $n$-groupoïde en termes d'un $(n-1)$-groupoïde (savoir son tronqué), \emph{muni} d'une $n$-gerbe sur le topos classifiant, liée par $\pi_n$ (``fordu'' bien sûr par l'action du $\pi_1$...).

\hangsection{Relativisation sur un topos.}\label{sec:app4}%
Bien sûr, il faut relativiser encore tout le yoga qu'on vient de décrire, au dessus d'un topos quelconque $X$. Il s'agirait donc de mettre en relation et d'identifier, dans un certaine mesure, d'une part l'algèbre homotopique sur $X$, d'autre part l'algèbre catégorique sur $X$ construite en termes de la notion de $n$-champ en groupoïdes ($n \geq 0$ fini ou infini). On espère que la notion d'image inverse de faisceau semi-simplicial par un morphisme de topos $f: X \to X'$ (qui est évidente) correspond à la notion évidente d'image directe de $n$-champs; et inversement, la notion évidente d'image directe de $n$-champs par $f$ devrait correspondre à une notion plus subtile d'image directe $\Lf_* (K_\bullet)$ d'un faisceau semi-simplicial, construit sansa doute dans l'esprit des foncteurs dérivés à partir de la notion naïve (mais on hésite s'il faut mettre $\Lf_*$ ou $\Rf_*$)... Les dévissages à la Postnikoff doivent avoir encore une interprétation remarquablement simple en termes de $n$-champs. Comparer à la remarque de Giraud qu'un 1-champ en groupoïdes sur $X$ peut s'identifier au couple d'un faisceau $\pi_0$ sur $X$, et d'une 1-gerbe sur le topos induit $X_{/\pi_0}$ (dont le lien, comme de juste, devrait être note $\pi_1$ !). D'ailleurs, dans le cas des 1-champs en groupoïdes, la traduction de ces animaux en termes de topos classifiants au dessus de $X$ est, je crois, développé en long et en large dans Giraud (il parle, si je me rappelle bien, d'``extensions'' du topos $X$). L'extension (si j'ose dire) de ce tapis aux $n$-champs ne devrait pas poser de problème.

Remords : tâchant de préciser heuristiquement la notion de topos classifiant d'un $n$-champ en groupoïdes (ou plus particulièrement, d'un $n$-groupoïde) pour $n \geq 2$, je vois que je n'y arrive pas à vue de nez. (Bien sûr, il suffirait (procédant de proche en proche) de savoir définir un topos classifiant raisonnable pour une \emph{$n$-gerbe}, liée par un faisceau abélien $\pi_n$). Donc je ne sais comment décrire le dévissage de Postnikoff en termes de $n$-champs, sauf pour $n \leq 2$. Ceci est lié à la question d'une description directe des groupes de cohomologie d'un $n$-groupoïde $C$ (ou d'un $n$-champ), à coefficients disons dans un système local commutatif, de fa\c{c}on que pour $C = C_n(K_\bullet)$, $K_\bullet$ un ensemble semi-simplicial dont les $\pi_i$ pour $i \geq n + 1$ sont nuls, on trouve les groupes de cohomologie correspondants de $K_\bullet$. Peut-on le faire en associant à $C$, de fa\c{c}on convenable, un ensemble semi-simplicial ``nerf'' de $C$ ?

Bien entendu, si on réussit à définir un topos classifiant pour $C$, celui-ci devrait être homotope à $K_\bullet$ ci-dessus, donc avoir les mêmes invariants homotopiques $\pi_i$ et cohomologiques $H^i$ ; itou pour les champs. La définition habituelle du topos classifiant, dans le cas $n = 1$, a bien cette vertu. Cas particulier typique de problème de la définition du topos classifiant : pour $\pi$ un groupe commutatif, trouver un topos canonique (fonctoriel en $\pi$ bien sûr...) ayant le type d'homotopie de $K(\pi, n)$, et qui généralise la définition du topos classifiant pour $n = 1$ (topos des ensembles où $\pi$ opère). On frémit à l'idée que les topos pourraient ne pas faire l'affaire, et qu'il y faille des ``$n$-topos'' !! (J'espère bien que ces animaux n'existent pas...)

\hangsection{Ingrédients principaux vers une ``Algèbre Topologique''.}\label{sec:app5}%
La théorie ``d'algèbre homologique non commutative'' que j'essaie de suggérer pourrait se définir, vaguement, comme l'étude parallèle des notions suivantes et de leurs relations des notions suivantes et de leurs relations multiples: a) espaces topologiques, topos, b) ensembles semi-simpliciaux, faisceaux semi-simpliciaux etc. c) $n$-catégories (notamment $n$-groupoïdes), $n$-champs (notament $n$-champs en groupoïdes) etc. d) complexes de groupes abéliens, de faisceaux abéliens. (Les ``etc'' réfèrent surtout aux structures supplémentaires qu'on peut envisager sur les objets du type envisagé...). C'est donc de l'algèbre avec la présence constante de motivations provenant de l'intuition topologique. Si une telle théorie devait voir le jour, il lui faudrait bien un nom, je me demande si ``algèbre topologique'' ne serait pas le plus adéquat (``algèbre homologique non commutative'' ne peut guère aller à la longue, pour des raisons évidentes). Ce qui est aujourd'hui parfois désigné sous ce [] n'est guère qu'un bric à brac de notions (telles que anneau topologique, corps topologique, groupe topologique etc) qui ne forment guère un corps de doctrine cohérent - il ne s'impose donc pas que cela accapare un nom qui servirait mieux d'autres usages. (Comparer le nouvel usage du terme ``géométrie analytique'' introduit par Serre, et qui ne semble guère avoir rencontré de résistance.)

Re-salut, et au plaisir de te lire

\bigskip

\par\hfill Villecun le \ondate{17.2.}1975\par

Cher Breen,

\hangsection{Champs essentiellement localement constants et types d'homotopie.}\label{sec:app6}%
Encore un ``afterthought'' à une lettre-fleuve sur le yoga homotopique. Comme tu sais sans doute, à un topos $X$ on associe canoniquement un pro-ensemble simplicial, donc un ``pro-type d’homotopie'' en un sens convenable. Dans le cas où $X$ est "localement homotopiquement trivial", le pro-objet associé est essentiellement constant en tant que pro-objet dans la catégorie homotopique, donc $X$ définit un objet de la catégorie homotopique usuelle, qui est son "type d’homotopie". De même, si $X$ est ``localement homotopiquement trivial en dim $\leq n$'', il définit un type d’homotopie ordinaire ``tronqué en dim $\leq n$'' - construction familière pour $i = 0$ ou $1$, même à des gens comme moi qui ne connaissent guère l’homotopie !

Ces constructions sont fonctorielles en $X$. D’ailleurs, si $f: X \to Y$ est un morphisme de topos, Artin-Mazur ont donné une condition nécéssaire et suffisante \emph{cohomologique} pour que ce soit une "équivalence d’homotopie en dim $\leq n$'' : c’est que $\mathrm H^i(Y, F) \tosim \mathrm H^i(X, f^*(F))$ pour $i \leq n$, et tout faisceau de groupes \emph{localement constant} $F$ sur $Y$, en se restreignant de plus à $i \leq 1$ dans le cas non commutatif. Ce critère, en termes de $n$-gerbes ``localement constantes'' $F$ sur $Y$, s’interprète par la condition que $F(Y) \to F(X)$ est une $n$-équivalence pour tout tel $F$ et $i \leq n$. Il est certainement vrai que ceci équivaut encore au critère suivant 
\begin{enumerate}
\item[(A)]\label{it:App6.A} Pour tout $n$-champ ``localement contant'' $F$ sur $Y$, le $n$-foncteur $F(Y) \to f^*(F)(X)$ est une $n$-équivalence;
\end{enumerate}
ou encore à
\begin{enumerate}
\item[(B)]\label{it:App6.B} Le $n$-foncteur $F \to f^*(F)$ allant de la $n$-catégorie des $(n-1)$-champs localement constants sur $Y$ dans celle des $(n-1)$-champs localement constants sur $X$, est une $n$-équivalence.
\end{enumerate}

En d’autres termes, les constructions sur un topos $X$ qu'on peut faire en termes de $(n-1)$-champs \emph{localement} constants ne dépendent que de son "(pro)-type d’homotopie $n$-tronquée", et le définissent. Dans le cas où $X$ est localement homotopiquement trivial en dim $\leq n$, donc définit un type d’homotopie $n$-tronqué ordinaire, on peut interpréter ce dernier comme un $n$-groupoïde $C_n$, (défini à $n$-équivalence près). En termes de $C_n$, les $(n-1)$-champs localement constants sur $X$ doivent s’identifier aux $n$-foncteurs de la $n$-catégorie $C_n$ dans la $n$-catégorie $(n-1)-\Cat$ de toutes les $(n-1)$-catégories. Dans le cas $n = 1$ ceci n’est autre que la théorie de Poincaré de la classification des revêtements de $X$ en termes du ``groupoïde fondamental'' $C_1$ de $X$. Par extension, $C_n$ mérite le nom de \emph{$n$-groupoïde fondamental de $X$}, que je propose de noter $\Pi_n(X)$. Sa connaissance induite donc celle des $\pi_i(X)$ $(0 \geq i \geq n)$ et des invariants de Postnikoff de tous les ordres jusqu’à $\mathrm H^{n+1}(\Pi_{n-1}(X), \pi_n)$.

Dans le cas d’un topos $X$  quelconque, pas nécessairement localement homotopiquement trivial en dim $\leq n$, on espère pouvoir interpréter les $(n-1)$-champs localement constants sur $X$ en termes d’un $\Pi_n(X)$ qui sera un pro-n-groupoïde. Ça a été fait en tous cas, plus ou moins, pour $n = 1$ (du moins pour $X$ connexe); le cas où $X$ est le topos étale d’un schéma est traité in extenso dans SGA 3\scrcomment{\textcite{SGA3}}, à propos de la classification des tores sur une base quelconque. 

Dans le cas $n = 1$, on sait qu'on récupère (à équivalence près) le $1$-groupoïde $C_1$ à partir de la $1$-catégorie $\bHom(C_1, \Ens)$ de ces foncteurs dans $\Ens = 0-\Cat$ (i.e. des ``systèmes locaux'' sur $C_1$ qui est un topos, dit "multigaloisien") comme la catégorie des ``foncteurs fibres'' sur le dit topos, i.e. la catégorie opposée à la catégorie des points de ce topos (lequel n’est autre que le \emph{topos classifiant} de $C_1$). Pour préciser pour $n$ quelconque la façon dont le $n$-type d’homotopie d’un topos $X$ (supposé localement homotopiquement trivial en dim $\leq n$, pour simplifier), i.e. son $n$-groupoïde fondamental $C_n$, s’exprime en termes de la $n$-catégorie des ``$(n-1)$-systèmes locaux sur $X$'' i.e. des $(n-1)$-champs localement constants sur $X$, et par là élucider complètement l’énoncé hypothétique $(B)$ ci-dessus, il faudrait donc expliciter comment un $n$-groupoïde $C_n$ se récupère, à $n$-équivalence près, par la connaissance de la $n$-catégorie $C_n = n-\bHom (C_n, (n-1)-\Cat)$ des $(n-1)$-systèmes locaux sur $C_n$. On aurait envie de dire que $C_n$ est la catégorie des ``$n$-foncteurs fibres'' sur $C_n$, i.e. des $n$-foncteurs $C_n \to (n-1)-\Cat$ ayant certaines propriétés d’exactitude (pour $n = 1$, c’était la condition d’être les foncteurs image inverse pour un morphisme de topos, i.e. de commuter aux $\varprojlim$ quelconques et aux $\varinjlim$ finies ...)  C’est ici que se matérialise la peur, exprimée dans ma précédente lettre, qu’on finisse par tomber sur la notion de $n$-topos et morphismes de tels ! $C_n$ serait un topos (appelé le "$n$-topos classifiant du $n$-groupoïde $C_n$), $(n-1)-\Cat$ serait le $n$-topos ``ponctuel'' type, et $C_n$ d’interprète modulo $n$-équivalence comme la $n$-catégorie des ``$n$-points'' du $n$-topos classifiant $C_n$. Brr !

Si on espère encore pouvoir définir un bon vieux $1$-topos classifiant pour un $n$-groupoïde $C_n$, comme solution d’un problème universel, je ne vois guère que le problème universel suivant : pour tout topos $T$, considérons $\bHom(\Pi_n(T), C_n)$. C’est une $n$-catégorie, mais prenons en la $1$-catégorie tronquée $\tau_1 \bHom(\Pi_n(T), C_n)$. Pour $T$ variable, on voudrait $2$-représenter le $2$-foncteur contravariant $\Top^{\circ} \to 1-\Cat$ par un topos classifiant $\B = \B_{C_n}$, donc trouver un $\Pi_n(\B) \xrightarrow{\phi} C_n$ $2$-universel en le sens que pour tout $T$, le foncteur
$$
\bHom_{\Top}(T, \B) \xrightarrow {u \mapsto \phi \circ \Pi_n(u)} \tau_1 \bHom(\Pi_n(T), C_n)
$$
soit une équivalence. Pour $n = 1$ on sait que le topos classifiant de $C_1$ au sens usuel fait l'affaire, mais pour $n = 2$ déjà, je doute que ce problème universel ait une solution. C'est peut-être lié au fait que le ``théorème de Van Kampen'', qu'on peut exprimer en disant que le $2$-foncteur $T \to \Pi_1(T)$ des topos localement $1$-connexes vers les groupoïdes transforme (à $1$-équivalence près) sommes amalgamées (et plus généralement commute aux $2$-limites inductives), n'est sans doute plus vrai pour le $\Pi_2(T)$. Ainsi, si $T$ est un espace topologique réunion de deux fermés $T_1$ et $T_2$, il n'est sans doute plus vrai  que la donnée d'un $1$-champ localement constant sur $T$ ``équivaut à'' la donnée d'un $1$-champ localement constant $F_i$ sur $T_i$ $(i = 1, 2)$ et d'une équivalence entre les restrictions de $F_1$ et $F_2$ à $T_1 \cup T_2$ (alors que l'énoncé analogue en termes de $0$-champs, i.e. de revêtements, est évidemment correct).

%App7
\hangsection{Intégration de champs et cohomologie.}\label{sec:app7}%
L'énoncé (B) plus haut rend clair comment expliciter la cohomologie d'un $n$-groupoïde $C_n$. Si $C_n = \Pi_n(X)$, et si $F$ est un $(n-1)$-champ localement constant sur $X$, $e^X_{n-1}$ est le $(n-1)$-champ ``final'', on a une $(n-1)$-équivalence de $(n-1)$-catégories
$$
\Gamma_X(F) = F(X) \simeq \bHom(e^X_{n-1}, F)
$$
qui montre que le foncteur $\Gamma_X$ ``intégration sur $X$'' sur les $(n-1)$-champs localement constants, qui inclut la cohomologie (non commutative) localement constante de $X$ en dim $\leq n-1$, s'interprète en termes de ``$(n-1)$-systèmes locaux'' sur le groupoïde fondamental comme un $\bHom(e^{C_n}_{n-1}, F)$ où maintenant $F$ est interprété comme un $n$-foncteur 
$$
C_n \xrightarrow{F} (n-1)-\Cat
$$
et $e^{C_n}_{n-1}$ est le $n$-foncteur constant sur $C_n$, de valeur la $(n-1)$-catégorie finale.

Pour interpréter ceci en notation cohomologique, il faut que j'ajoute, comme ``remords'' à la lettre précédente, l'interprétation explicite de la cohomologie non commutative sur un topos $X$, en termes d'intégration de $n$-champs sur $X$. Soit $F$ un $n$-champ de Picard strict sur $X$, il est donc défini par un complexe de cochaines $L'$ sur $X$
$$
0 \to L^0 \to L^1 \to L^2 \to ... \to L^n \to 0
$$
concentré en degrés $0 \leq i \leq n$ (défini à isomorphisme unique près dans la catégorie dérivée de $\Ab(X)$). Ceci dit, les $\mathrm H^i(X, L')$ (hypercohomologie) \emph{pour $0 \leq i \leq n$} s'interprètent comme $\mathrm H^i(X, L') = \pi_{n-i}\Gamma_X(F)$.

Si on s'intéresse à tous les $\mathrm H^i$ (pas seulement pour $i \leq n$) on doit, pour tout $N \geq n$, regarder $L'$ comme un complexe concentré en degrés $0 \leq i \leq N$ (en prolongeant $L'$ par des $0$ à droite).Le $N$-champ de Picard strict correspondant n'est plus $F$ mais $C^{N-n}F$, où $C$ est le foncteur ``espace classifiant'', s'interprétant sur les $n$-catégories de Picard strictes comme l'opération consistant à ``translater'' les $i$-objets en des $(i+1)$-objets, et à rajouter un unique $0$-objet; il se prolonge aux $n$-champs de Picard ``de fa\c{c}on évidente'', on espère, de fa\c{c}on à commuter aux opérations d'image inverse de $n$-champs. On aura donc pour $i \leq N$
$$
\mathrm H^i(X, L') = \pi_{N-i}\Gamma_X(C^{N-n}F) \quad i \leq N.
$$
Ceci posé, il s'impose, pour tout $n$-champ de Picard strict $F$ sur $X$, de poser 
$$
\boxed{
\mathrm H^i(X, F) = \pi_{N-i}\Gamma_X(C^{N-n}F) \quad~\text{si}~\quad N \geq i,n
}
$$
ce qui ne dépend pas du choix de l'entier $N \geq Sup(i, n)$ [{\textbf{NB}} On a un morphisme canonique de $(n-1)$-groupoïdes,
$$
C(\Gamma_X F) \to \Gamma_X(C F),
$$
comme le montrent les constructions évidentes en termes de complexes de cochaines, et on voit de même que celui-ci induit des isomorphismes pour les $\pi_i$ pour $1 \leq i \leq n+1$.]

{\textbf{NB}} On voit en passant que pour un $n$-champ en groupoïdes $F$ sur $X$, si on se borne à vouloir définir les $\mathrm H^i(X, F)$ pour $0 \leq i \leq n$, on n'a pas besoin sur $F$ d'une structure de Picard, car il suffit de poser
$$
\mathrm H^i(X, F) = \pi_{n-i}(\Gamma_X(F)) \quad 0 \leq i \leq n.
$$
Si d'autre part $F$ est un $n$-$\Gr$-champ (i.e. muni d'une loi de composition $F \times F \to F$ ayant les propriétés formelles d'une loi de groupe) le $(n+1)$-``champ classifiant'' est défini, et on peut définir $\mathrm H^i(X, F)$ pour $i \leq n+1$ par
$$
\mathrm H^i(X, F) = \pi_{n+1-i}(\Gamma_X(CF))
$$
en particulier
$$
\mathrm H^{n+1}(X, F) = \pi_0(\Gamma_X(CF)) =~\text{sections de}~CF~\text{à équivalence près}.
$$

Mais on ne peut former $CCF = C^2F$ et définir $\mathrm H^{n+2}(X, F)$, semble-t-il \emph{que} si $CF$ est lui-même un $\Gr$-$(n+1)$-champ, ce qui ne sera sans doute le cas que si $F$ est un $n$-champ de Picard strict...

Venons en maintenant au cas où $F$ est un $n$-champ \emph{localement constant} sur $X$, donc défini par un $(n+1)$-foncteur
$$
C_{n+1} \xrightarrow{F} n-\Cat.~\text{de Picard strictes}.
$$
Alors, posant pour $0 \leq i \leq n$
$$
\mathrm H^i(C_ {n+1}, F) = \pi_{n-1}(\bHom(e_n^{C_{n+1}}, F)),
$$
``on a fait ce qu'il fallait'' pour que l'on ait un isomorphisme canonique
$$
\mathrm H^i(C_{n+1}, F) \simeq \mathrm H^i(X, F),
$$
(valable en fait sans structure de Picard sur $F$...). Il s'impose, pour tout $\infty$-groupoïde $C$ et tout $(n+1)$-foncteur
$$
C \xrightarrow{F} n-\Cat.~\text{de Picard strictes}.
$$
de définir les $\mathrm H^i(C, F)$, pour tout $i$, par
$$
\mathrm H^i(C, F) = \pi_{N-i}\bHom(e_N^C, C^{N-n}F)
$$
où on choisit $N \geq Sup (i, n)$. Si $F$ n'a qu'une $\Gr$-structure (pas nécessairement de Picard) on peut définir encore les $\mathrm H^i(C, F)$ pour $i \leq n+1$ par
$$
\mathrm H^i(C, F) = \pi_{n+1-i}\bHom(e_{n+1}^C, CF).
$$
Dans le cas $C = C_{n+1} = \Pi_{n+1}(X)$, il doit être vrai encore (en vertu de (A) plus haut), que cet ensemble est canoniquement isomorphe à $\mathrm H^{n+1}(X, F) = \pi_0 \Gamma_X(CF)$ (c'est vrai et bien facile pour $n = 0$). Décrire la flèche canonique entre les deux membres de 
$$
\mathrm H^{n+1}(X, F) \simeq \mathrm H^{n+1}(\Pi_{n+1}X, F) \quad ?
$$
Si on veut réexpliciter (A) et (B), en termes du yoga (C), on arrive à la situation suivante:

On a un $(n+1)$-foncteur entre $(n+1)$-groupoïdes
$$
f_{n+1}: C_{n+1} \to D_{n+1}
$$
induisant par troncature un $n$-foncteur
$$
f_n: C_n \to D_n
$$
On doit avoir alors:
\begin{enumerate}
    \item[(A')] $f_n$ est une $n$-équivalence si et seule si le $n$-foncteur $\phi \to \phi \circ f_n$
    $$
    f_n^*: \bHom(D_n, (n-1)-\Cat) \to \bHom(C_n, (n-1)-\Cat)
    $$
\end{enumerate}
allant des $(n-1)$.systèmes locaux sur $D_n$ (ou $D_{n+1}$, c'est pareil) vers les $(n-1)$-systèmes locaux sur $C_n$, est une $n$-équivalence.
\begin{enumerate}
    \item[(B')] $f_n$ est une $n$-équivalence si et seule si pour tout $n$-système local $F$ sur $D_{n+1}$,
    $$
    F: D_{n+1} \to n-\Cat,
    $$
    le $n$-foncteur induit par $f_{n+1}$
    $$
    \underbrace{\bHom(e_n^{D_{n+1}}, F)}_{\hspace*{-5mm}\Gamma_{D_{n+1}(F)}\hspace*{-5mm}}\; \to \underbrace{\bHom(e_n^{D_{n+1}}, f^*_{n+1} F)}_{\hspace*{-5mm}\Gamma_{C_{n+1}(F)}\hspace*{-5mm}}\;
    $$
    est une $n$-équivalence.
\end{enumerate}

\hangsection{Les trois approches vers la cohomologie d'un topos.}\label{sec:app8}%
La construction de la cohomologie d'un topos en termes d'intégration des champs ne fait aucun appel à la notion de complexe de faisceaux abéliens, encore moins à la technique des résolutions injectives. On a l'impression que dans son esprit, via la définition (qui reste à expliciter !) des $n$-champs, elle s'apparenterait plutôt aux calculs ``Cechistes'' en termes d'hyperrecouvrements. Or ces derniers se décrivent à l'aide d'une petite dose d'algèbre semi-simpliciale. Si oui, cela ferait essentiellement trois approches distinctes pour construire la cohomologie d'un topos : 
\begin{enumerate}
\item[a)] point de vue des complexes de faisceaux, des résolutions injectives, des catégories dérivées (\emph{algèbre homologique commutative});
\item[b)] point de vue Cechiste ou semi-simplicial (\emph{algèbre homotopique});
\item[c)] point de vue des $n$-champs (algèbre catégorique, ou \emph{algèbre homologique non-commutative}).
\end{enumerate}
Dans a) on ``résoud'' les coefficients, dans b) on résoud l'espace (ou topos) de base, et dans c) en apparence on ne résoud ni l'un ni l'autre. 

Bien cordialement,

\bigskip

\par\hfill Villecun \ondate{17/19} July 1975\par

Dear Larry,

\hangsection{Question about a non-abelian Dold-Kan theorem.}\label{sec:app9}%
I am happy to finish by receiving an echo to my long letter and even a beginning to a constructive approach to a theory of the type I envisaged. The construction which you propose for the notion of a non-strict $n$-category, and of the nerve of the functor, has certainly the merit of existing, and of being a first precise approach, but otherwise can be subject to some evident criticism: it is very technical, unintuitive (yet at the level of $1-\Cat$, etc, and even of $2-\Cat$, everything is so clear ``you just follow your nose...''). And finally the absence of a definition of a functor sending (semi-)simplicial sets to $n$-groupoids. This functor correspond to a geometric intuition so clear that a theory which does not include it seems to me kind of a joke! Perhaps in trying to write down (like a sort of list of Christmas presents!) in a complete and explicit enough way the notions which one would like to have at ones disposal, and the relations (functor, equivalence, etc.) which should link them, one would arrive finally at a kind of axiomatic description sufficiently complete which should either give the key to a explicit \emph{ad hoc} construction, or should permit at least to enunciate and prove a theorem of existence and uniqueness\footnote{As was seen in section 9, ``uniqueness'' here has to b understood in a considerably wider sense than I expected, when writing this letter to Larry Breen. It now appears that the whole theory of stacks of groupoids will depend on the choice of a ``coherator'' $\bC$, as seen in section 13.} for a theory of the required type.

Otherwise, not having understood the idea of Segal in your last letter (which I have generously sent to Illusie\dots), I do not see how you define the Picard $n$-categories - but this matters little. As far as ``strict'' Picard $n$-categories are concerned, all I ask of them is that they finally form an $(n + 1)$-category $(n + 1)$-equivalent to that of chain complexes of length $n$. Agreed? I thank you for having rectified in my mind a big blunder, due to my great ignorance of algebraic topology and homotopy - I was in fact of the impression that $H$-spaces satisfying conditions of associativity and commutativity strict enough (say equivalent to an $\Omega^i X$ with $i$ arbitrarily large) correspond to commutative topological groups (inspired by several analogies\dots). Thus I am entirely in agreement with your observations on p.5.

On the other hand, I am still intrigued by the following question: is there an analogue of the ``tapis'' of Dold-Puppe\footnote{Tim Porter pointed out to me that ``Dold-Puppe'' is an inaccuracy name for this basic theorem, which should be called \emph{Dold-Kan theorem.}} for semi-simplicial groups (\emph{not necessarily commutative}) and what form should it take? To tell the truth I consider the yoga 
$$
\text{simplicial sets} \leftrightarrow \infty-\text{groupoids}
\leqno{(*)}
$$
as being essentially the ultimate ``set theoretic'' version of Dold-Puppe, which I would deduce from (*) by making explicit solely the fact that the abelian groups in $\infty-\Cat$ are ``nothing else'' than the chain complexes in $\Ab$. One should therefore first determine what should be the groups in $\infty-\Cat$. I can tell you what these are in $1-\Cat$ . this will be discussed at length in the book of Mme. Sinh\scrcomment{\textcite{GCS}}, I think in the chapter ``\emph{strict} $\Gr$-categories'' (i.e. the isomorphisms of associativity, for unity and inverse $X X^{-1} \simeq 1$ are \emph{identities}). One can make explicit for example how (\emph{via} the fact that a $\Gr$-category is $\Gr$-equivalent to a strict $\Gr$-category) the calculation with the $\Gr$-categories reduces to a very algebraic calculation with the \emph{strict} $\Gr$-category, by a kind of ``calculus of fractions'' (by choice, left or right) of the type which is used in giving the construction of derived categories. In any case, here is the explicit formulation of the structures (groups in $1-\Cat$) in terms of the theory of groups ($1$-categories in $\Gr$\footnote{AS was pointed out to me by Ronnie Brown, this structure was already well-known to J.H.C. Whitehead, under the name of ``crossed module'', and extensive use and extensive generalizations of this notion (in quite different directions from those I was having in mind, in terms of $\Gr$-stacks over an arbitrary topos) have been made by him and others. With respect to the question on next page, of generalizing this notion of ``non-commutative chain complex'' from length one to length two, Ronnie says there is a work in preparation by D. Conduché ``Modules croisés généralisés de longueur 2''.}). The structure is described by a quadruplet $(L_1, L_0, d, \theta)$ with 
$$
L_1 \xrightarrow{d} L_0
$$
a homomorphism of ordinary groups, 
$$
\theta: L_0 \to \Aut_{\Gr}(L_1)
$$
an operation of $L_0$ on $L_1$, with the following two axioms:
\begin{enumerate}
    \item[(a)] $d$ commutes with the operation of $L_0$, when $L_0$ acts on $L_1$ via $\theta$ and on itself by inner automorphisms:
    $$
    d(\theta(x_0) x_1) = \text{int}(x_0) d(x_1)
    $$
    \item[(b)] $\theta (d()x_1) = \text{int}(x_1)$.
\end{enumerate}
These properties imply that $\text{Im}d$ is normal in $L_0$ (hence $\Coker d = \pi_0$ is defined) and $\pi_1 = \Ker d$ is central in $L_1$, and finally that $L_0$ operates on $L_1$ leaving $\pi_1$ invariant, and it operates \emph{via} $\pi_0$. The principal cohomological invariant of this situation is evidently the Postnikoff-Sinh invariant
$$
\alpha \in \mathrm H^3 (\pi_0, \pi_1).
$$
I have met these animals - without even looking for them - in many situations, which I will not list now (I came across them recently \emph{a propos} the classification of ``ordinary'' formal groups over a perfect field, in terms of \emph{affine} algebraic groups, and \emph{commutative} formal groups, related by the strict $\Gr$-structures of this type (except that one has to use this formalism in an arbitrary topos (not merely in $\Sets$)) - to make explicit the yoga that ``the transcendent character of a formal group is concentrated essentially in the commutative formal groups'', discovered it seems by Dieudonné\dots). The question which I wish to raise is the generalisation to groups in $n-\Cat$, where I expect to find a non-commutative chain complex
$$
L_n \to L_{n-1} \to \dots \to L_1 \to L_0 \to 1
$$
with supplementary structures doubtless of the type of $\theta$, but what are they? It is understood that the topological significance of such structures is that they express exactly the ``truncated homotopy type in dim $\leq n$'' of topological groups, or equivalently the homotopy type in dim $\leq n + 1$ of pointed connected topological spaces\dots). Have you candidates to propose ?

\starsbreak

\hangsection{The ``six operations'' and homology.}\label{sec:app10}%
Your reflections on biduality and homology, however formal, tie in with a crowd of developments, of which only some exists at present, and others would demand considerable work still. Here are the reminiscences which your naive questions bring to mind:
(A) The formalism of the $\Rf_!$, $\Rf^!$ (combined with $\Rf_*$, $\Lf^*$, $\Lotimes$ and $\RHom$, ``the six operations'') carries implicitly in itself the definition of homology and the essential identity between homology and cohomology. One now has this formalism for quasi-coherent sheaves on schemes - seminar Hartshorne (Springer L.N. 20) - for the topological spaces and arbitrary sheaves of coefficients - Verdier, \emph{exposé} Bourbaki (SNLM 300) - and for the étale cohomology of schemes for ``discrete'' coefficients (``$\ell$-adic'' or torsion) prime to the residual characteristic (SGA 5), finally, for coherent sheaves on analytic spaces (Verdier-Ruget). (The formalism remains to be developed in the crystalline context, and in the characteristic 0 in the context of stratified modules with singularities, à la Deligne, with perhaps - over the field $\mathbf{C}$ - the introduction of additional Hodge structures, finally in the context of motives; I am convinced that it exists about anywhere - maybe, wherever there is a formalism of a cohomological nature.)
    
    Working in étale cohomology on a separated scheme of finite type over a field $k$, say, with a ring of coefficients $\Lambda$ of torsion prime to the characteristic, the complex of sheaves $f^!(\Lambda_e)$ (where $\hat{f}: X \to \Spec k = e$) plays the role of \emph{complex of singular chains on $X$ with coefficients in $\Lambda$}, and $\Rf_!$ $(f^! \Lambda_e)$ plays the role of a \emph{homology $\mathrm{H}_*(X/e)$}, \emph{vis a vis} of course, of coefficients on $e$ which are complexes of $\Lambda$-modules. You can easily justify this assertion with the help of the ``global duality theorems'', by one or two tricks which I spare you here.

{\textbf REMARKS}.
\begin{enumerate}
    \item[(1)] There is no need to truncation, it works in all dimensions.
    \item[(2)] This is related (at least as far as the philosophy is concerned) to he fact that for the various types of coefficients (under conditions of ``constructibility'') one has a theorem of ``biduality'', at least if one allows resolution of singularities (but Deligne has told me I believe that he knows a proof without that), with values in a ``dualizing complex'' $K_e$ (on $e$), $K_X$ (on $X$). If for example $\Lambda$ is ``self-dualising'' (or Gorenstein) for example $\Lambda = \mathbf{Z}/n \mathbf{Z}$, one can take $K_e = \Lambda$, therefore the dualising complex $K_X = f^! (K_e)$ is nothing else than the ``complex of singular chains with coefficients in $\Lambda$.
    \item[(3)] One can do the same thing for coefficients such as $\mathbf{Z}_{\ell}$ (Jouanolou, thesis non published\scrcomment{\textcite{JOU69}}, I fear!)
    \item[(4)] This works also for $f: X \to S$ finitely presented separated if $f$ has the properties of ``cohomological local triviality'' (properties ``local upstairs'') for example $f$ \emph{smooth}; one finds that $\mathrm{H}_*(X/S) = \Rf_! f^!(\Lambda_S)$.
\end{enumerate}

\starsbreak

(B) Artin-Mazur have studied in a spirit close to yours the \emph{autoduality} of the Jacobian of a relative curve $X/S$. It is necessary to ask them for precise results, perhaps it works say if $X/S$ is proper and flat or relative dimension 1 - in any case it is OK on a discrete valuation ring with smooth \emph{generic} fibre. The special fibre could be very wild. (I have used their results in SGA 7 to prove, in the case of Jacobians, a duality conjecture on the group of connected components associated to the Neron models of abelian varieties dual one to another\dots). 
Towards the end of the 50's (beginning of 60's?), when the grand cohomological stuff ($f^!$, $f^!$, étale cohomology, etc.) just came out from darkness, the course given by Serre on the theory of Rosenlich and Lang on generalised jacobians and the geometric class field theory (see Serre's book) and later the ``geometric'' theory of \emph{local} class field theory making use of pro-algebraic groups (see his article on this subject), made me reflect on the cohomological formulations of these and other results, which should be of a ``geometric'' nature, such that the ``arithmetic'' results over an arbitrary base field (or residue field) $k$ (finite, for example) follow immediately by descent from the ``geometric'' case of base field $\overline{k}$. I exchanged letters with Serre - I don't know if I can find copies - but I recall that I sketched projects for some ambitious enough theories on generalised residues, generalised local jacobians, etc.,  in at least three different directions. But I have never, in spite of numerous attempts, succeeded in mobilising someone for developing one of these programmes. Here a few words on them: 

\starsbreak

\hangsection{Complex of generalised jacobians.}\label{sec:app11}%
(C) In the situation where $X$ is of finite type over a \emph{field} $k$, construction of a complex of generalised jacobians $J_{* X/k}$ (of length equal to dim $X$).
    
    This is a complex of affine commutative pro-algebraic groups on $k$, with the exception of $J_0$ if I remember well, ($J_0$ had as abelian part the abelian part of $\Alb_{X/k}$, the usual generalised jacobian). It's construction, inspired by the residual complex, passes by generalised jacobians (in an appropriate cohomological sense) of the localisation $\Spec_{\mathbf{O}_{X, x}}$ of $X$ at its different point.
N.B. $\mathbf{H}_O (J_*)$ was the ``generalised Jacobian'' of $X$, i.e. there existed a homeomorphism $X \to \mathbf{H}_0$, which was universal for homomorphisms of $X$ into commutative locally proalgebraic groups. For $X$ connected, $\mathbf{H}_O$ is an extension of $\mathbf{Z}$ by an appropriate proalgebraic group. It is possible that, at first, I restricted to the case of $X$ smooth.

The cohomology role of this complex was that of a complex of \emph{homology}
$$
\mathrm{H}^i(X, G_X) \simeq \Ext^i(J_{* X/k}, G)
\leqno{(*)}
$$
but for which coefficients? I believe I took arbitrary commutative algebraic groups $G$ but worked with the Zariski topology (malédiction !). Even in the case of discrete $G$, I considered the Zariskian $\mathrm{H}^i$, this gives slightly stupid cohomology groups, evidently. I realised that one should work ultimately in étale cohomology, and that the construction of the $(J_i)_{X/k}$ will evidently be modified accordingly. As for the significance of the $\Ext^i$ (hypercohomology), at a moment where Serre had developed the formalism for proalgebraic groups, one was not too fearful of taking it in the category of such objects - and in the sense of a ``derived category'' which at that moment had never yet been explicitly defined and studied. (We have, after all, somewhat progressed since those days!). I have the impression, in view of these antique cogitations, heuristic as they were, that it should now be possible to develop at present such a theory of $J_{* X/k}$, in cohomology fppf, giving a formula (*) without limitation on the degree $i$ of the cohomology. (N. B. But $J_*$ evidently no longer stops in dim $X = n$ but in dim $2n$. It is nevertheless possible that the components $J_i$ might be of dim 0 for $i > n$).

I believe that the construction of the $J_*$ does not commute with base change, but merely does so in the derived category sense.

\starsbreak

\hangsection{Global ``geometric'' class field theory as a cohomological duality formula. Serre duality and the ``Lang trick''.}\label{sec:app12}%
(D) Let $X/k$ be a smooth scheme (for simplicity) over a field $k$, separated and of finite type, or relative dimension $d$, and $n$ an integer $> 0$. If $n$ is prime to the characteristic and if $F$ is a sheaf of coefficients on $X$ which is annihilated by $n$, the global duality tells us that $\Rf_!(F)$ and $\Rf_*(\RHom(F, \mu_n^{\otimes d}))$ ($\mu_n$ = sheaf of n-th roots of unity = $\Ker(G_m \xrightarrow{n} G_m)$) are dual to each other with values in $(\mathbf{Z}/n\mathbf{Z})_k$, for example $\Rf_!(\mathbf{Z}/n\mathbf{Z})$ and $\Rf_*(\mu_n^{\otimes d})$, or $\Rf_!(\mu^{\otimes}_m)$ and $\Rf_*(\mathbf{Z}/n\mathbf{Z})$, are dual to each other - at least with a shift of amplitude $2d$ in dimension. (As $\mathbf{Z}/n\mathbf{Z}$ is injective over itself, this gives in fact perfect duality
    $$
    \mathrm{R}^if_!(F) \times \mathrm{R}^{2d-i}f_*(\RHom(F, \mu_n^{\otimes d})) \to \mathbf{Z}/n\mathbf{Z}.)
    $$
    If now one no longer assumes $n$ prime to he characteristic, for example $n$ is a power of $p$ = characteristic of $k > 0$), it seems that everything collapses: to start with, one no longer knows (for $d > 1$) by what to replace $\mu_n^{\otimes d}$\dots
The extraordinary miracle is that for $d = 1$, i.e. $X$ a smooth curve, everything continues to work perfectly, provided one states things with care! The first verifications are made for example with $F = \mathbf{Z}/p\mathbf{Z}$, $\mu_p$, or $\alpha_p$, with $X$ complete - one finds it's O.K. by virtue essentially of the autoduality of the jacobian. One can make these examples more sophisticated on taking \emph{twisted} coefficients, and $X$ not complete - one convinces oneself this works always! Simply, it is necessary to note that here the $\mathrm{R}^if_*(F)$, $\mathrm{R}^if_!(F)$ have a ``continuous'' structure (they are essentially poalgebric groups). This corresponds to the well known phenomenon in class field theory that the structure of $\pi_{1\text{ab}}$ of $X$, when $X$ is not complete, is \emph{continuous} - hence same holds for $\mathrm{H}^1(X, \mathbf{Z}/p^n\mathbf{Z})$ say.

By the way, I point out for you that Serre once proposed (without ever writing it down, I think) a theory of duality for \emph{commutative unipotent} algebraic groups, \emph{modulo radical isogeny}, duality with values in $\mathbf{Q}/\mathbf{Z}$ (or $\mathbf{Q}_p/\mathbf{Z}_p$). He found that if (when $k$ is algebraically closed, say) $G$ is such a group, then $G' = \Ext^1(G, \mathbf{Q}/\mathbf{Z})$ can canonically be given a structure of quasi-algebraic group (i.e. defined modulo radical isogeny), doubtless in a unique manner provided it verifies some functorial properties, and on requiring that for $G = \mathbf{G}_a$ one finds that $\Ext^1(\mathbf{G_a}, \mathbf{Q}/\mathbf{Z}) \simeq \mathbf{G}_a$ with the usual structure. Let $\Delta G = G' = \Ext^1(G, \mathbf{Q}/\mathbf{Z})$. One finds $G \simeq \Delta\Delta G$ i.e. $\Delta$ is an authentic autoduality! I call $\Delta$ \emph{Serre duality}. It surely goes over to ind-progroups on an arbitrary base field (not necessarily algebraically closed) in the case $p > 0$. Moreover, for finite étale groups, it is $\Ext^0(G, \mathbf{Q}/\mathbf{Z})$ (pontrjagin duality) which gives a perfect duality. One could screw together, in an appropriate derived category, Serre duality and Pontrjagin duality, by taking $G \mapsto \Delta G = \RHom(G, \mathbf{Q}/\mathbf{Z})$: one calls this (``cohomological'') Serre duality. This will be a magnificent autoduality, if one puts oneself in a derived category where the $\mathbf{H}^i$ of the envisaged complexes are (up to passing to the limit) extensions of étale groups by connected unipotent groups. Now one gets only such complexes, by ``integrating'' finite coefficients $F$ on $X$ by $\Rf_!$ or $\Rf_*$. This being said, by passing to the limit in the initial formulation (or equivalently by replacing the $(\mathbf{Z}/n\mathbf{Z})_k$, previously considered, by $(\mathbf{Q}/\mathbf{Z})_k$ on $k$, and forming $f^!(\mathbf{Q}/\mathbf{Z})_k = (\mu_\infty)_X$) the duality formula takes the form
$$
\Delta(\Rf_!(F)) \simeq \Rf_*(\text{DF}[2]) \quad \text{``shift'' of dimension}
$$
where $D$ is the ``Cartier duality'' $\RHom(F, \mu_\infty)$ (or $\RHom(F, \mathbf{G}_m)$ if one prefers?), and $\Delta$ is the Serre duality: cohomology with proper supports and with arbitrary supports are exchanged by duality, when one takes upstairs Cartier duality, and downstairs Serre duality.

The validity of the duality formula is not open to doubt - the principal work for establishing it consist certainly in a careful description of the category of coefficients with which one is working, as well on $X$ as on $k$, and of the functors $D$ and $\Delta$. As the definition of an arrow is immediate, once the building of the machine has been accomplished, the validity of the formula should result without difficulty from the usual ``dévissages'' which allow one to verify the duality in the particular standard cases $F = \mathbf{Z}/p\mathbf{Z}$, $\mu_p$, $\alpha_p$ on a smooth, complete $X$. (N.B. the case of coefficients prime to the characteristic is already known.) Let us make explicit what the formula of duality says for $\mathrm{R}^1f_*(\mathbf{G}_X)$, where $G$ is a finite group étale on $k$ (the most important case being $G = (\mathbf{Z}/p^m\mathbf{Z})_k$); one recovers Serre's description of ``geometric class field theory'' in terms of extensions by $G$ of a generalised jacobian of $X$. Thus, the duality formula can be understood as a cohomological version, considerably enriched, of geometric class field theory. When the base field $k$ is finite, to retrieve the class field theory in the classical form, one can use ``the trick of Lang'' (on the relation between the ``arithmetic'' $\pi_1$ of a smooth, connected commutative algebraic group $J$ on $k$ and its $\mathrm{H}^0(k, J) = J(k)$: the $\pi_1^{\text{ar}}(J)$ classifies the isogenies above $J$ with kernel a constant group $\pi_1^{\text{ar}}(J) \simeq \mathrm{H}^0(k, J)$) - i its cohomological form, which may be stated: 
$$
\Delta_0\RGamma_K(J^*) \simeq \RGamma_K(\Delta J^*[1]),
$$
where $\Delta$ is Serre duality, $\Delta_0$ Pontrjagin duality for the totally disconnected topological abelian groups (duality with values in $\mathbf{Q}/\mathbf{Z}$), $J^*$ a complex of algebraic ind-progroups on $k$. Taking account of this ``Lang duality formula'' and applying $\RGamma_K$ to the formula of duality for geometric class fields, one gets the ``duality formula of arithmetic class field theory'':
$$
\Delta_0 (\mathrm{H}_!(X, F)) \simeq \mathrm{H}^*(X, D(F) [3])
$$
(isomorphism of totally disconnected topological groups).

Another remark: when $F$ is not an ``étale sheaf'', but has a continuous structure such as $\alpha_p$, one must be careful in the definition of $\Rf_!(F)$, for $X$ non complete, starting from the compactification $\widetilde{X}$; thus, if $F$ comes from an ``admissible'' sheaf $\widetilde{F}$ on $\widetilde{X}$, one must have an exact triangle
$$
\begin{tikzcd}[row sep=tiny]
  & \Rf_!(\hat{\widetilde{F}}) \ar[dl] \\ \Rf_!(F) \ar[rr] & & \mathrm{R}\widetilde{f}_*(\widetilde{F}) \ar[ul],
\end{tikzcd}
$$
where $\hat{\widetilde{F}}$ is the \emph{formal completion} of $\widetilde{F}$ along $\widetilde{X} - X$ (a finite number of points\dots). It is here, unless I am mistaken, that appears the link with local class field theory, in its cohomological version, on which I am going now to say a few words.

\starsbreak

\hangsection{Case of local ``geometric'' class field theory.}\label{sec:app13}%
(E) \textbf{Local class field theory as a duality formula}
    
    Let $V$ be a complete discrete valuation ring with residue field $k$ - assume either that $k$ has been lifted to $k \subset V$ (and therefore $V \simeq k[[T]]$) or that $k$ is perfect of characteristic $p > 0$. In order to fix ideas, and to be sure that I'm on solid ground, I consider at first on $K$ (= the field of fractions of $V$) \emph{finite} coefficients $F$ (as on $X$ previously) and I consider the objects $\mathrm{H}^1(K, F)$, or $\RGamma_K(F)$. The main work to be done consists in defining an adequate category of coefficients over $k$ (perhaps the same one as in (D)) and a functor
    $$
    F \mapsto \mathrm R \underline{\Gamma}_K(F)
    $$
    with values in the category of such coefficients, in such a manner that the following isomorphisms holds.
    $$
    \RGamma_K(F) \simeq \RGamma(\mathrm R \underline{\Gamma}_K(F)).
    $$
    This correspond to the intuition (acquired directly from elementary examples) according to which for $k$ algebraically closed, say, the $\mathrm{H}^0(K, F)$, $\mathrm{H}^1(K, F)$\dots are endowed with a structure of $k$-algebraic group (ind-pro\dots). In this construction, the ring scheme of Witt vectors over $k$ (introduced by Serre) and the ``Greenberg functor'' (associating to a $V$-scheme a $k$-prescheme) will play an essential role.
    
This being done, the duality formula will be formally stated as in (D) above: 
$$
\Delta \mathrm R \underline{\Gamma}_K(F) \simeq \mathrm R \underline{\Gamma}_K(\text{DF}[1])
$$
where $D$ stands for Cartier duality, $\Delta$ for Serre duality. When the residue field is finite, it becomes (via ``Lang's trick'' mentioned previously)
$$
\Delta_0 \RGamma_k(F) \simeq \RGamma_K(\text{DF}[2])
$$
$\Delta_0$ standing for Pontrjagin duality. The formula contains local geometric class field theory à la Serre, and arithmetical local class field theory in its classical form.

\textbf{Remarks}.

(a) If $F$ is prime to the residue characteristic the formula is very easy to prove and well known. It may be considered a very special case of the ``induction formula'' for a morphism $i: s \mapsto S$, in the duality formalism:
$$
i^!(D_S(F)) = D_S(i^*(F))
$$
(we take here the inclusion of $p = \Spec(k)$ in $S = \Spec(V)$). Thus the work to be done concerns the $p$-primary coefficients, for $p =$ characteristic $k > 0$. The most subtle case is that of unequal characteristic.

(b) The functor $\RGamma$ may be obtained by composing $\mathrm{R}j_*$ (where $j: U = \Spec(K) \to \Spec(V) = S$ is the inclusion) with a cohomological version of the ``Greenberg functor''.

(c) In (D) and (E), I restricted myself to finite coefficients $F$ - it's for those that I am sure of what I assert. But it is certainly true that the duality formula is even richer, that something may still be asserted for example for $F$ a not necessarily finite group scheme, for example an abelian scheme (with a few degenerate fibres in the case of (D)?), but I have never entirely clarified this question, even on a heuristic basis. I vaguely recall a formula which should be contained in the formalism (say if $k$ is algebraically closed): for $F$ an abelian scheme on $K$, $F$ the dual abelian scheme and $G'$ the pro-algebraic group over $k$ attached ``à la Greenberg'' to its Néron model, then one has
$$
\mathrm{H}^1(K, F) \stackrel{?}{\simeq} \Ext^1_{k-\text{grp}}(G', \mathbf{Q}/\mathbf{Z})
$$
(N.B. without any guarantee.) In principle, the previously mentioned duality conjecture concerning Néron models of SGA 6 should come out of the local duality machine.

(d) You may ask Deligne if he didn't dive into questions (D) and (E) lately.

\hangsection{Comments on fppf cohomology versus crystalline cohomology.}\label{sec:app14}%
(F) \textbf{Significance and limitations of the fppf topology}
    
    Since the attempts of Serre to find a ``Weil chomology'' by using the cohomology of a scheme with coefficients not only discrete $\mathbf{Z}/p^n\mathbf{Z}$ ($n \to \infty$) or $\mu_{p^n}$ ($n \to \infty$), but also continuous (for example $W_n$, $n \to \infty$), which give good results for recovering a correct $\mathrm{H}^1$, during numerous years I have come upon the impression, which I have tried in vain to make precise, that a correct ``$p$-adic'' ``Weil cohomology'', in the case $p > 0$ and $k$ of characteristic $p$, should come, in one way or another, from the fppf cohomology, for finite coefficients for example, or more general coefficients, e.g. algebraic groups over $k$. The construction in (B) of the local jacobian complex was, of course, related to this hope: the homology might reveal what is hidden to us in cohomology! For some time now, one has at ones disposal the formalism of crystalline cohomology, and one knows (Berthelot) that (at least for $X$ projective and smooth) it has the correct properties. If one uses that as a kind of standard by which to ``measure'' the other cohomologies, one finds that the part of the crystalline cohomology $\mathrm{H}^i_{\text{cris}}(X)$ which could be described in terms of fppf cohomology of $X$ with coefficients in algebraic $k$-groups is a small part of $\mathrm{H}^i$ only; more precisely, using the very rigid supplementary structure of the $\mathrm{H}^i$ (modules of finite type on the ring $W(k)$ of Witt vectors) which comes from the existence of the Frobenius homomorphism (an isogeny), $\mathrm{H}^i \xrightarrow{F} \mathrm{H}^i$ (semi-linear), one finds that one keeps always in the part ``of slope $\leq 1$ (although the possible slopes vary between 0 and $i$\dots). This explains why for $i = 1$ one can obtain via fppf a correct $\mathrm{H}^i$, although for $\mathrm{H}^2$ already all the attempts have been unfruitful.
In truth, one conjectures that \emph{all} the part of slope $\leq 1$ in $\mathrm{H}^i_{\text{cris}}$ comes from fppf. But I have completely lost contact with these questions - people such as Mazur, Kats, Messing - and of course Deligne - should be knowledgeable as to the present states of these questions.

\starsbreak

\hangsection{The homotopical trinity: ``spaces'', ``$\infty$-groupoids'', (essentially locally constant) ``coefficients categories''\dots}\label{sec:app15}%
Your question 7 seems to indicate that there is a misunderstanding on your part on the significance of the ``homotopy type'' of $X$, for $X$ a topos (for example the étale topos of a scheme). Doubtless you must be confusing the homotopical algebra which one can perform on $X$, using semi-simplicial sheaves, stacks of all kinds, the relations between these - and the other point of view according to which $X$ (with its very rich structure of topos) virtually disappears so as to become no more than a pale element of a ``homotopical category'' (or pro-homotopical), deduced from the topos by a very thorough process of ``localisation''. At first sight, all that still remains with poor stripped $X$, are the $\pi_i$ - and its cohomology groups with constant coefficients - or at the worst twisted constant coefficients. When one digs more into this definition of ``what is left to this poor $X$'' one falls precisely on \emph{the locally constant $n$-stacks} (as an $f: X \to X'$ which is a homotopy equivalence induces a $(n + 1)$-equivalence between the categories of locally constant $n$-stacks on $X$ and on $X'$) - which of course contain the abelian chain complexes of length $n$ of sheaves with locally constant cohomology sheaves, and the hyper-cohomology of these. It is thus that one arrives at this triangle of objects which mutually determine each other
\[\begin{tikzcd}[column sep=-3em,every arrow/.append style={<->}]
 & \begin{tabular}{@{}c@{}}
    topos (or topological space\\
    or semi-simplical complex)\\
    \textit{modulo} $n$-homotopy
   \end{tabular}
   \arrow[dl]\arrow[dr]&\\
 \begin{tabular}{@{}l@{}}
    $n$-groupoids\\
    (up to $n$-equivalence)
   \end{tabular}
   \arrow[rr]&
 & \begin{tabular}{@{}l@{}}
    ``special'' $n$-categories\\
    (up to $n$-equivalence)
   \end{tabular}
\end{tikzcd}
\]
One says that an $n$-category $E_n$ is ``special'' (or \emph{$n$-galois}) if it is $n$-equivalent to the category of locally constant $(n-1)$-stacks on an appropriate topological space (or a topos), or, what should be equivalent, if it is $n$-equivalent to the category of $n$-functors $G_n \to n-\Cat$, where $G_n$ is an $n$-groupoid. If $X$, $G_n$, $E_n$ correspond in this way, one calls $G_n$ \emph{the fundamental $n$-groupoid} of $X$, or of $E_n$, or says that $E_n$ is \emph{the category of local $(n-1)$-systems} on $X$, or on $G_n$, or that $X$ is \emph{the geometric realisation} of $G_n$ or of $E_n$. In analogy with the familiar case $n = 1$, it should be possible to interpret $G_n$ as the full sub-$n$-category of $\Hom_n(E_n, n-\Cat)$ formed by the $n$-functors $E_n \to n-\Cat$ satisfying certain exactness properties (one feels like saying: which commute with finite $\varprojlim$ and arbitrary $\varinjlim$); but this raises the disquieting vision of $n$-limits in $n$-categories. (N. B. The case $n = 2$ begins to become familiar to us\dots). It is prudent in all of this to suppose that $X$ is ``locally homotopically trivial'', which ensures the pro-simplicial set which Artin-Mazur associate to it (with the help of nerves of hyper-coverings) is essentially constant in the ordinary homotopy category - thus $X$ defines a homotopy type in the usual sense. This is surely \emph{not} the case for the étale topos of a scheme. In such case, the fundamental $n$-groupoid should be conceived as a \emph{pro-$n$-groupoid} (nothing surprising in that, in view of the familiar theory of $\pi_1$), and $E_n$ as an (ind)-$n$-category (the ind-structure will correspond to the exigencies of local triviality for a variable $n$-stack, relative to coverings more and more fine on $X$).

\starsbreak

\hangsection{The ``six operations'' for $P$-constructible sheaves (for a given equi-singular stratification $P$).}\label{sec:app16}%
I nevertheless understand your instinctive resistance to conceive this extreme stripping of a beautiful topos $X$, to the point of retaining only the meagre homotopy type. Even more, I am persuaded that going to the root of this instinctive resistance, one arrives at a generalisation and deepening of the notion of ``homotopy type'', and to bring new grist to the mill of the development of a good homotopical yoga. Here is what I have in mind.

Let us speak first of sheaves (of sets, or of modules, etc.) instead of stacks, for simplicity, and place ourselves in the étale topos of a scheme. The locally constant sheaves - modulo a supplementary condition of finiteness which is sufficiently anodyne - form the easiest of the \emph{constructible} sheaves, for the definition of which they serve as models. Supposing $X$ coherent (= quasi-coherent and quasi-separated), then the general constructible sheaves are those for which there exists a finite partition $X = \cup_{i \in I} X_i$ of $X$ into ``cells'' or ``strata'' $X_i$, each locally closed and constructible, such that the restriction of $F$ to every $X_i$ is locally constant (also a finiteness condition\dots). Thus the category of constructible sheaves on $X$ (which gives back the category of all sheaves on passing to a category of ind-objects\dots) may itself be thought of as an inductive limit of categories associated to finer and finer partitions on $X$. One can then, for such a fixed partition $P$, set out to study the category of sheaves (or complexes of sheaves, or stacks) which are ``$P$-constructible'' (or, more generally, which are ``locally constant'' on every $X_i$). These categories will not have truly satisfying structures unless they are stable for the usual operations - such as $\RHom$, or $\mathrm{R}j_* j^*$ where $j: X_i \to X$ is a ``cell'' of the partition, etc. In fact, if $X$ is excellent and one has resolution of singularities at ones disposal, one knows that the torsion constructible sheaves (under the proviso of being prime to the characteristic) are stable for all these operations - but not for a finite partition of $X$ fixed once and for all. To have such a finer stability, it is necessary to make some very strict hypotheses of \emph{``equi-singularity''} on the given stratifications of $X$, along the strata. I think nonetheless that a refinement of known techniques will show that $X$ admits arbitrarily fine stratifications having these properties of equi-singularity (and with the $X_i$ regular and connected, but this does not matter for our present purpose).

By way of example, suppose that there are just two strata, the closed one $X_0$, and $X_1 = X \textbackslash X_0$. According to Artin's devissage, giving oneself a sheaf $F$ on $X$ is equivalent to giving a sheaf $F_0 = i^*_0(F)$ on $X_0$, a sheaf $F_1(= i_1^* F)$ on $X_1$, and a homomorphism $F_0 \to i^*_0 i^*_{1^*}(F_1) = \varphi(F_1)$, where $i_0$, $i_1$ are the inclusions $X_0 \xrightarrow{i_0} X \xleftarrow{i_1} X_1$. In order that $F$ should be $P$-constructible, it is necessary and sufficient that $F_0$ and $F_1$ should be locally constant (plus some accessory finiteness conditions\dots), on $X_0$ and $X_1$ respectively. Then (by virtue of the hypothesis of equi-singularity) the same will be true of $\varphi(F_1)$, and the category of sheaves in which we are interested can be expressed entirely in terms of the category of locally constant sheaves on $X_0$ and $X_1$, i.e. of the mere homotopy type of $X_0$ and $X_1$, except that we must make explicit the nature of the left exact functor $\phi$. I think tht this should be possible, in the context of \emph{schemes} in which I am placed (technically rather sophisticated), on introducing an ``étale tubular neighbourhood'' of $X_0$ in $X_1$ (which is a very interesting topos, but not associated to a scheme). But this technical construction is only a paraphrase of an extraordinary simple topological intuition, which I will make explicit, supposing, to fix the ideas, that the base field is $\mathbf{C}$ and so one may work with locally compact spaces in the usual sense. The topological idea behind the hypothesis of equi-singularity  that there exists a \emph{tubular neighbourhood} $T$ of $X_0$ in $X$ retracting onto $X_0$ and such that the pair $(X_0, T)$ over $X_0$ should be a locally trivial bundle, i.e. that $T \textbackslash X_0$ is locally trivial over $X_0$. In fact if $\partial T$ is the ``boundary'' of $T$, which also should be a locally trivial bundle on $X_0$, then $T$ over $X$ is the conic bundle (= bundle where fibres are cones) ($\simeq (\partial T \times I) \amalg_{\partial T} X_0$ where $I = [0, 1]$, $\partial T \to \partial T \times I$ is defined by $x \mapsto (x, 1)$, and $\partial T \to X_0$ is the projection) then $\overset{\circ}{T} = T \textbackslash X_0 \simeq \partial T \times [0, 1 [$ is $X_0$-homotopic to $\partial T$. If $X_0$ and $X_1$ are non singular, then so also will be $\overset{\circ}{T}$ and $\partial T$, which are then topologically smooth fibrations on $X_0$. Moreover, putting $\widetilde{T}_1 = X_1 \textbackslash \overset{\circ}{T}$, the inclusion $\widetilde{X}_1 \to X_1$ is a homotopy equivalence, and $X$ can be recovered, up to homeomorphism, from the diagram of spaces
\[\begin{tikzcd}
  (\overset{\circ}{T} = T \textbackslash X_0 \simeq) \partial T \ar[r,swap,"\text{inclusion}","j"']\ar[d,swap,"\text{fibration}","p"'] & \widetilde{X}_1(\simeq X_1) \\
  X_0 &
\end{tikzcd}\]
as an amalgamated sum. In terms of this diagram of spaces, the above functor $\phi$ interprets immediately as
$$
\phi(F_1) \simeq p_* j^*(\widetilde{F}_1)
$$
where $F_1 \to \widetilde{F}_1$ is the restriction from $X_1$ to $\widetilde{X}_1$ (which is an equivalence of categories for the envisaged (locally constant) sheaves). Giving $F = (F_0, F_1, u:F_0 \to \phi F_1)$ can then also be made explicit as giving
$$
F_0, \widetilde{F}_1, \widetilde{u}: p^*(F_0) \to j^*(\widetilde{F}_1)
$$
where $F_0(\widetilde{F}_1)$ are locally constant sheaves on $X_0$ (respectively $X_1$). It is necessary to recall that here $p$ is a real fibration, and $j$ is an inclusion (in practice, for the case $X_0$, $X_1$ smooth, the inclusion of the boundary in a manifold with boundary).

If you prefer, one can also take the diagram which is less pretty (but a little more canonical)
\[\begin{tikzcd}
  \overset{\circ}{T} \ar[r,"j'"]\ar[d,swap,"p'"] & X_1 \\
  X_0 &
\end{tikzcd}\]
coming essentially to the same thing, as it is formed from spaces homotopic to the preceding one. One can even replace $X_0$ by $T$ ($X_0$ being a deformation retract of it) and write 
\[\begin{tikzcd}
  \overset{\circ}{T} \ar[r]\ar[d,swap,"p''"] & X_1 \\
  T &
\end{tikzcd}\]
where ``literally'' $p''$ is now an inclusion, but ``morally'', it is a \emph{fibration} with very pretty fibres (notably compact of finite dimension, and moreover non-singular varieties - this is much better than that which is given by the yoga of Cartan-Serre ``every continuous mapping is equivalent to a fibration''\dots). This last diagram however has the advantage of being amenable to a purely algebraic, direct construction, in the context of schemes, once one has developed the construction of étale tubular neighbourhoods\footnote{Tim Porter pointed out to me that work on étale tubular neighbourhoods was done by D.A. Cox: ``Algebraic tubular neighbourhoods I, II'', Math. Scand. 42 (1978) 211-228, 229-242. I've not seen yet this work, and can't say therefore whether it meets the rather precise expectations I have for a theory of tubular neighbourhoods, for the needs of a dévissage theory of stratified schemes (or, more generally, stratified topoi)}.

\hangsection{Relation with ``indexed homotopy types'' and with ``dévissage'' of stratified spaces. The ``fine homotopy type'' of a (tame) space or of a scheme.}\label{sec:app17}%
The point I wish to come to, is that the consideration f sheaves (or complexes thereof, or $n$-stacks\dots) which are $P$-constructible on an $X$, where $P$ is a given ``equi-singular'' stratification, reduces in our particular cases to the knowledge of a diagram of ordinary \emph{homotopy types} (or pro-types, if one comes back to the étale topology)
\[\begin{tikzcd}
  \Delta \ar[r,"j"]\ar[d,swap,"p"] & X_1 \\
  X_0 &
\end{tikzcd}\]
by taking local coefficients systems (or locally constant $n$-stacks) on the vertices $X_0$, $X_1$, which are related to each other by a homomorphism of compatibility of the type $p^*(F_0) \to j^*(F_1)$. It should be an amusing exercise (which I have not yet done) to verify and to make explicit how the ``six operations'' on sheaves (either on $X$, or on a subspace which is a union of strata of $X$) can be expressed in this dictionary, in the case, let us say, of non-singular strata (otherwise, there will be a difficulty with the dualising complexes, which one would prefer to have as objects in our category), and to reestablish the known formulae involving these operations. But it appears probable that, to carry out this transcription well, it would be necessary, rather than considering a diagram of type
\[\begin{tikzcd}
  \bullet \ar[r]\ar[d] & \bullet \\
  \bullet &
\end{tikzcd}\]
in the homotopical category formed from the category of semi-simplicial sets, to consider the category of diagrams of semi-simplicial sets, and to pass from these to the homotopical category of fractions\footnote{This is the typical game embodied in the ``derivator'' associated to the theory $\Hot$ of usual homotopy types (compare section 69).}

I have recently more or less made explicit, while thinking on the foundations of ``tame topology'', (i.e. where one eliminates from start all wild phenomena) how an equi-singular stratification, say with non singular strata, of a compact ``tame space'', gives rise canonically to a diagram of space which are manifolds with boundary, the arrows of the diagram being essentially locally trivial fibrations of manifolds with boundary on the others (with fibres which are compact manifolds with boundary), \emph{and} the inclusion of the boundary in these manifolds with boundary (in fact, one finds slightly more general inclusions, certain boundaries which appear being endowed with an ``elementary'' cellular decomposition, i.e. the closed strata are again manifolds with boundary which are glued together along common parts of the boundary; and it is also necessary to consider the inclusions of these pieces one in another\dots), and can be reconstituted from this diagram by gluing\footnote{Some more details on this program are outlines in \emph{``Esquisse d'un Programme''} (section 5), in \emph{Réflexions Mathématiques 1}}. In other words, one has a canonical devissage description of tame compact spaces $X$, eventually endowed with equi-singular stratifications with non-singular strata, in terms of finite diagrams of a precise nature made out o manifolds with boundary. When we are interested in sheaves (or complexes of sheaves, or $n$-stacks) which are $P$-constructible on $X$, where $P$ is such a fixed stratification, these may be described in terms of the envisaged diagram, of which only the ``homotopy type'' is to be retained. One foresees that the six operations on these sheaves can be translated in an ad hoc manner to this homotopical context. Finally, if instead of having only one compact tame space $X$, one has, let us say, a tame morphism $f: X \to Y$ of such objects, then by choosing equi-singular stratifications on $X$ and $Y$ adapted to $f$ (the strata of $X$ being in particular locally trivial fibrations on those of $Y$\dots), one should find a ``morphism'' from the diagram of manifolds with boundary expressing $X$ int that expressing $Y$ (with natural morphisms which essentially reduce to fibrations of compact manifolds with boundaries on other such objects) in such a way that the four operations $\Rf_*$, $\Rf_!$, $\mathrm{L}g^*$, $g^!$ between $P_{X^-}$ and $P_{Y^-}$ constructible sheaves on $X$ and $Y$ (or on locally closed sub-spaces $X'$, $Y'$ which are union of strata, such that $f$ induces $g: X' \to Y'$) can be expressed in terms of standard operations between the mere homotopy types. Finally, all these constructions, still partially hypothetical (there is work on the foundations to be done!) should be able to be paraphrased in the framework of excellent schemes, by making use of the machinery of étale tubular neighbourhoods. In one or other case, the ``fine homotopy type'' of a tame space, respectively of an excellent scheme, is defined by passage to the limit from ``$P$-homotopy type'' associated to finer and finer equi-singular stratifications $P$ (with non-singular strata).

This ``fine'' homotopy type would embody the knowledge, not only of sheaves or locally constant $n$-stacks, but (via a passage to the inductive limit) the knowledge of \emph{all of them}. And it would depend, in a suitable sense, functorially on $X$. In the case of a scheme of finite type on an algebraically closed field $k$ say, the strongest cohomological and homotopical \emph{finiteness theorem} would be expressed precisely in terms of a fine homotopy type, and would say that \emph{the ordinary homotopy types which are their constituents are essentially ``finite polyhedra''} - and even compact manifolds with boundary - or in more precise fashion, their profinite completions (in the sense of Artin-Mazur) prime to the characteristic $p$ of $k$ are those of such polyhedra. One sees clearly how to begin on such programme in characteristic 0, but one foresees supplementary amusement, or even mystery, in the case $p > 0$, for the varieties which, even birationally, resist being lifted to characteristic 0!

\starsbreak

\hangsection{How far do ``essentially constant'' \emph{abelian} coefficients determine a homotopy type?}\label{sec:app18}%
From these essentially geometric thoughts, I could not at this moment draw up a precise programme for developing adequate algebraic structures to express them. I restrict myself to several marginal remarks.

For a long time I have been intrigued by the idea of a ``linearisation'' of an (ordinary) homotopy type, i.e. questions of the type: if $X$ is a homotopy type, how much cohomological information of the type: cohomology of $X$ with variable coefficients $M$ (constant or twisted constant), multiplicative structure $\mathrm{H}^i(X, M) \times \mathrm{H}^j(X, N) \to \mathrm{H}^{i + j}(X, M \otimes N)$, then eventually other cohomology operations - is it necessary to have to reconstruct entirely the homotopy type? (say, in this preliminary pre-derived category approach, assuming given the fundamental group $\pi_1$, and therefore the category of constant twisted coefficients (= $\pi_1$-modules), the functors $\mathrm{H}^i(-, M)$ over these, together with the structure of cohomological functors relative to exact sequences, the structure of cup-product, etc. - related by certain formal properties?) Once one has at one's disposal the language of derived categories: the sub-category of the derived category of abelian complexes on $X$, formed from complexes the sheaves of cohomology of which are locally constant on $X$, with its triangulated structure and its multiplicative structure $\Lotimes$ (and eventually $\RHom$\dots) gives a more satisfying candidate for hoping to recover the homotopy type. I don't really know if this suffices the recovery indeed\footnote{I was informed by knowledgeable people soon later that the answer is well known to be negative, by working with ``rational homotopy types'' (the cohomology of which is made up with vector spaces over $\mathbf{Q}$). It is well known indeed that a 1-connected rational homotopy type is \emph{not} known from its rational cohomology ring alone, which contains already all the information I was contemplating. At last this is so if we assume that $\mathrm{H}^i(X)$ is of finite dimension over $\mathbf{Q}$ for all $i$. But is there a counterexample still when $X$ is a homotopy type ``of finite type''?}, but on the other hand I have no doubt that on pursuing ``linearisation'' to the end, that is to say by going to the \emph{non-abelian} framework, and working with the $(n + 1)$-category (without any supplementary structure on it!) of locally constant $n$-stacks of constructible sheaves on $X$, for all $n$, one manages to reconstruct the homotopy type via its fundamental $\infty$-groupoid, as explained in my previous letter and recalled in this one. (This signifies in particular that all the possible and imaginable cohomology operations are already included in the data furnished by such a system of $n$-categories\dots).

Similarly, the more elaborate homotopy type, which are related to certain finite diagrams, which one can associate to certain types of stratification $P$ of tame topological spaces $X$, let's say, should correspond in as perfect a fashion to the $(n + 1)$-category of $n$-stacks on $X$ which are locally constant on each of the strata of $P$ (say: which are subordinated to $P$). If the above description of homotopy types by the ``locally constant derived category'' was valid indeed, one would expect to recover here the mixed homotopy type from the corresponding sub-category of the derived category of abelian sheaves on $X$, provided by the complexes which have locally constant cohomology on each of the strata - with also the operations $\Lotimes$, $\RHom$, plus in case of need, the four operations $\mathrm{R}g_!$, $\mathrm{R}g_*$, $\mathrm{L}g^*$, $g^!$ for the induced $g: Z' \to Z''$ of the various locally closed unions of strata\dots The problem here is that we don't at present even know what is a triangulated category, not any more than what is its non-commutative version, described probably more simple and more fundamentally: a ``homotopical category'' with operations of taking ``fibres'' and ``cofibres''\footnote{This ``problem'' is met with by the notion of a ``derivator'', which ``was in the air'' already by the late sixties, but was never developed (instead even derived categories became tabu in the seventies\dots).}.

It is surely time that I finish this ``lettre-fleuve'', which is becoming more and more vague. Just one question: what is this marvellous formula of Bloch-Quillen to which you allude, of which I have never heard, and which makes my mouth water?

\bigskip

Very cordially yours, 

\begin{flushright}
Alexandre Grothendieck
\end{flushright}











%End
